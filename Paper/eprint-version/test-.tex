% !TEX root =main.tex

\subsection{Evaluation}\label{full-Evaluation}

In this section, we provide a comparison summary between  SO-PoR and outsourced PoRs  \cite{armknecht2014outsourced,xu2016lightweight,Storage-Time}.   Note, there are two protocols  proposed in  \cite{Storage-Time}; In the following, we only consider the one that supports public verifiability, e.g. basic PoSt.  In our analysis, we consider a generic setting where a client outsources $z$ verifications. We outline the property and cost comparison results in Tables \ref{table::O-PoR-Property} and \ref{table::O-PoR-Cost}, respectively. Also, in our cost analysis, we compare SO-PoR cost with the cost of the most efficient privately verifiable traditional PoR \cite{DBLP:conf/asiacrypt/ShachamW08}, too. A full analysis is provided in Appendix \ref{Full-Evaluation}.

% !TEX root =main.tex


 \begin{table*}[htb]
%\begin{footnotesize}
\begin{center}
\caption{ \small{O-PoR's Properties Comparison. $\checkmark^{*}$ indicates the property is met only in theory.}} \label{table::O-PoR-Property} 
%\renewcommand{\arraystretch}{.9}
%\scalebox{0.98}{
\begin{tabular}{|c|c|c|c|c|c|c|c|c|c|c|c|c|c|} 
   \hline
\cellcolor[gray]{0.9}&
 \multicolumn{3}{c|}{\cellcolor[gray]{0.9}\scriptsize  \underline{\ \ \ \ \ \ \ \ \ \ \ \ \ \ \ \ \ \ \ \ \ \ \ \ \ \ \ \ \ \ \ \ \ \ \ \ \ \ \ \ \ \ \ \ \ \   \ \ \ \ \ Properties   \ \ \ \ \ \ \ \ \ \ \ \ \ \ \ \ \ \ \ \ \ \ \ \ \ \ \ \ \ \ \ \ \ \ \ \ \ \ \ \ \ \ \ \   \ \ \ }}\\
 %\cline{2-5}
 \cellcolor[gray]{0.9}\multirow{-2}{*} {\scriptsize Protocols}&\cellcolor[gray]{0.9}\scriptsize Real-time Detection&\cellcolor[gray]{0.9}\scriptsize Fair Payment&\cellcolor[gray]{0.9}\scriptsize Untrusted Auditor\\
\hline
 \cellcolor[gray]{0.9}{\scriptsize  SO-PoR }&\scriptsize$\checkmark$&\scriptsize$\checkmark$&\scriptsize$\checkmark$\\
    
     \cline{2-4}    
     \hline 
     
          \hline 
  \cellcolor[gray]{0.9}{\scriptsize   \cite{armknecht2014outsourced}}&\multirow{2}{*}{\rotatebox[origin=c]{0}{\scriptsize }} \scriptsize $\times$&\scriptsize$\times$&\scriptsize$\checkmark$\\
     \cline{2-4}
      
      \hline
      
       \hline
      
\cellcolor[gray]{0.9}{\scriptsize   \cite{xu2016lightweight}}&\multirow{2}{*}{\rotatebox[origin=c]{0}{\  \scriptsize }} \scriptsize$\times$&\scriptsize$\times$&\scriptsize$\times$\\
     \cline{2-4}


%%%%%%%%%%%%%%%%%%%%%%%
      \hline
      
       \hline
      
\cellcolor[gray]{0.9}{\scriptsize   \cite{Storage-Time}}&\multirow{2}{*}{\rotatebox[origin=c]{0}{\  \scriptsize }} \scriptsize$\times$&\scriptsize$\checkmark^{*}$&\scriptsize$\checkmark^{*}$\\
     \cline{2-4}


%%%%%%%%%%%%%%%%%%%%%%
 \hline
\end{tabular}
%}
\end{center}
%\end{footnotesize}
\end{table*}


% !TEX root =main.tex


 \begin{table*}[!htbp]

\caption{ \small Outsourced PoR's Cost  Comparison. In the table, $z$ is the total number of verifications, $c$ is  the number of challenges for each verification, $n$ is the total number of file blocks, $c'=(0.1)c$,  and $||\bm{F}||$ is a file bit size.} \label{table::O-PoR-Cost} 
\begin{footnotesize}
\begin{center}
\renewcommand{\arraystretch}{.80}
%\scalebox{1}{
%\begin{subtable}{.56\linewidth}%xxxx
\begin{minipage}{1\linewidth}
%\caption{\small Computation and I/O Costs}
\setlength{\tabcolsep}{1pt}%compress the table horizontally
\begin{tabular}{|c|c|c|c|c|c|c|c|c|c|c|c|c|c|c|c|} 

   \hline
\cellcolor[gray]{.9}&\cellcolor[gray]{.9}&
 \multicolumn{4}{c|}{\cellcolor[gray]{.9}\scriptsize \underline{ \  \  \  \  \   \  \    \  \  \  \ \  \  \  \  \  \  \ \  \  \      Computation Cost     \  \  \  \  \   \  \    \  \  \  \ \  \  \  \  \  \  \ \  \  \   }}& \multicolumn{4}{c|}{\cellcolor[gray]{.9}\scriptsize \underline{\  \  \  \  \  \  \ \ \  \  \  \  \  \   \  \  \  \  \  \  \  \  Communication Cost   \  \  \  \  \   \  \    \  \  \  \ \  \  \  \  \  \  \ \  \  \   }}\\
% \cline{3-6}
  \cellcolor[gray]{.9}\multirow{-2}{*}{\scriptsize Protocols} &\cellcolor[gray]{.9}\multirow{-2}{*} {\scriptsize Operation}&\cellcolor[gray]{.9}\scriptsize$\mathtt{Store}$&\cellcolor[gray]{.9}\scriptsize$\mathtt{SolvPuz}$&\cellcolor[gray]{.9}\scriptsize$\mathtt{Prove}$&\cellcolor[gray]{.9}\scriptsize$\mathtt{Verify}$&\cellcolor[gray]{.9} {\scriptsize Client}&\cellcolor[gray]{.9} {\scriptsize Server}&\cellcolor[gray]{.9} {\scriptsize Verifier}&\cellcolor[gray]{.9}{\scriptsize Proof Size}
  \\
\hline
    %SO-PoR 1st row
\cellcolor[gray]{.9}& \multirow{2}{*}{\rotatebox[origin=c]{0}{\cellcolor[gray]{.9}\scriptsize }} \scriptsize Exp.&\scriptsize$z+1$&\scriptsize$T z$&\scriptsize$-$ &\scriptsize$-$&\scriptsize $128(n+$&&&\\
     \cline{2-6}  
     %SO-PoR 2nd row
 \multirow{-2}{*}{\rotatebox[origin=c]{0}{\cellcolor[gray]{.9}\scriptsize  SO-PoR }}&\cellcolor[gray]{.9}\scriptsize Add. or Mul.&\scriptsize$2(n+cz)$ &\scriptsize$z$&\scriptsize$4 c z$&\scriptsize$2z(1+c)$&\scriptsize  $cz+19z)$&\multirow{-2}{*}{\scriptsize $884z$}&\multirow{-2}{*}{\scriptsize$-$}&\multirow{-2}{*}{\scriptsize $O(1)$}\\
     \cline{2-6}   
      
     \hline 
       %[3] 1st row 
       
          \hline 
          
 \cellcolor[gray]{.9}  &\multirow{2}{*}{\rotatebox[origin=c]{0}{\scriptsize }}\cellcolor[gray]{.9}\scriptsize Exp.&\scriptsize$9 n$&\scriptsize$ -$&\scriptsize$-$&$-$&&\scriptsize$256z+$&\scriptsize $4672n+$&\\
     \cline{2-6}
     %[3] 2nd row 
\cellcolor[gray]{.9}   \multirow{-2}{*}{\rotatebox[origin=c]{0}{\scriptsize   \cite{armknecht2014outsourced}}}  &\cellcolor[gray]{.9}\scriptsize Add. or Mul.&\scriptsize$10 n$&$-$&\scriptsize$4 z(c+c')$&\scriptsize$z(9c+3)$&\multirow{-2}{*} {\scriptsize $128n$}&\scriptsize $||{\bm{F}}||$&\scriptsize $256z$&\multirow{-2}{*} {\scriptsize $O(1)$}\\   
      
      \hline
      
       \hline
      %[53] 1st row
  \cellcolor[gray]{.9}    &\multirow{2}{*}{\rotatebox[origin=c]{0}{\  \scriptsize }}\cellcolor[gray]{.9}\scriptsize Exp.&\scriptsize$-$&$-$&\scriptsize$z (3+c)$&\scriptsize$6 z$&&&&\\
     \cline{2-6}
     %[53] 2nd row
 \cellcolor[gray]{.9}&\cellcolor[gray]{.9}\scriptsize Add. or Mul.&\scriptsize$4 n$&\scriptsize$-$&\scriptsize$2 z(3 c+4)$&\scriptsize$2 c z$&&&&\\ 
    \cline{2-6}
    %[53] 3rd row 
\cellcolor[gray]{.9}\multirow{-3}{*}{\rotatebox[origin=c]{0}{\scriptsize   \cite{xu2016lightweight}}}&\cellcolor[gray]{.9}\scriptsize Pairing&\scriptsize$-$&\scriptsize$-$&\scriptsize$7 z$&$-$&\multirow{-3}{*} {\scriptsize $2048n$}&\multirow{-3}{*} {\scriptsize $6144z$}&\multirow{-3}{*} {\scriptsize$-$}&\multirow{-3}{*} {\scriptsize $O(1)$}\\ 
 \hline
  
  %%%%%%%%%%%%%%%%%%%%%%%%%%%%%%
 
 \hline
\cellcolor[gray]{.9}& \multirow{2}{*}{\rotatebox[origin=c]{0}{\cellcolor[gray]{.9}\scriptsize }} \scriptsize Exp.&\scriptsize$-$&\scriptsize$3T z$&\scriptsize$-$ &\scriptsize$3z$&&\scriptsize$128cz \log n+$&&\\
     \cline{2-6}  
 \multirow{-2}{*}{\rotatebox[origin=c]{0}{\cellcolor[gray]{.9}\scriptsize  \cite{Storage-Time} }}&\cellcolor[gray]{.9}\scriptsize Add. or Mul.&\scriptsize$-$ &\scriptsize$Tz$&\scriptsize$-$&\scriptsize$-$&\multirow{-2}{*} {\scriptsize $128$}&\scriptsize $4096z$&\multirow{-2}{*} {\scriptsize$-$}&\multirow{-2}{*} {\scriptsize$O(\log n)$} \\
     \cline{2-6}    
     \hline 
 
 %%%%%%%%%%%%%%%%%%%%%%%%%%%%%%
 
 
 
\end{tabular}  %xxxxxx
\end{minipage}
%\end{center}
%\end{footnotesize}
%\end{table*}
%\end{subtable}


%\begin{subtable}{.52\linewidth}%xxxx
%\renewcommand{\arraystretch}{1.42}
%
%\begin{minipage}{.8\linewidth}
%\caption{\small Communication Cost (in bit)}
%\setlength{\tabcolsep}{.65pt}%compress the table horizontally
%
%\begin{tabular}{|c|c|c|c|c|c|} 
%   \hline
% \cellcolor[gray]{.9}\scriptsize Protocols&
%\cellcolor[gray]{.9}\scriptsize Client&\cellcolor[gray]{.9}\scriptsize Server&\cellcolor[gray]{.9}\scriptsize  Verifier&\cellcolor[gray]{.9}\scriptsize Proof Size\\
%   \hline
%   
%
%\cellcolor[gray]{.9}&\scriptsize$128 (n+$&\multirow{2}{*} {\scriptsize$884 z$} &\multirow{2}{*} {\scriptsize $-$}&\multirow{2}{*}{\scriptsize$O(1)$}\\  
%\cellcolor[gray]{.9}\multirow{-2}{*} {\scriptsize  SO-PoR}&\scriptsize$cz+19 z)$&&&\\
% \hline  
% 
%      \hline   
%\cellcolor[gray]{.9}& \multirow{2}{*} {\scriptsize$128 n$}&\multirow{2}{*} {\scriptsize$||\vv{\bm{F}}||+256 z$}&\scriptsize$4672 n+$&\multirow{2}{*}{\scriptsize$O(1)$}\\
%\cellcolor[gray]{.9}\multirow{-2}{*} {\scriptsize  \cite{armknecht2014outsourced}}&&&\scriptsize$256 z$&\\
% \hline
% 
%      \hline 
%\cellcolor[gray]{.9}\scriptsize\cite{xu2016lightweight}& \scriptsize$2048 n$&\scriptsize$6144z$&\scriptsize$-$&\scriptsize$O(1)$\\
% \hline
% 
%%%%%%%%%%%%%%%%%%%%
%
%
%      \hline 
%\cellcolor[gray]{.9}\scriptsize\cite{Storage-Time}& \scriptsize$128$&\scriptsize$(128cz\log n)+4096z$&\scriptsize$-$&\scriptsize$O(\log n)$\\
% \hline
%
%%%%%%%%%%%%%%%%%
% 
% 
%% \end{footnotesize}
%%\end{center} 
%\end{tabular}  %xxxxxx
%\end{minipage}
%\end{subtable}
%}
\end{center}
\end{footnotesize}
\end{table*}



%--------------------------------------------------------


% \begin{table*}[!htbp]
%\begin{footnotesize}
%\begin{center}
%\caption{ \small Outsourced PoR Cost  Comparison} \label{table::O-PoR-Cost} 
%\renewcommand{\arraystretch}{.80}
%%\scalebox{1}{
%%\begin{subtable}{.56\linewidth}%xxxx
%\begin{minipage}{1\linewidth}
%\caption{\small Computation and I/O Costs}
%\setlength{\tabcolsep}{.55pt}%compress the table horizontally
%\begin{tabular}{|c|c|c|c|c|c|c|c|c|c|c|c|c|c|c|c|} 
%
%   \hline
%\cellcolor[gray]{.9}&\cellcolor[gray]{.9}&
% \multicolumn{4}{c|}{\cellcolor[gray]{.9}\scriptsize \underline{\  \  \  \  \  \  \ \ \  \  \  \  \  \      Algorithms' Computation Cost     \  \  \  \ \  \  \  \  \  \  \ \  \  \   }}&\cellcolor[gray]{.9}&\cellcolor[gray]{.9}&\cellcolor[gray]{.9}&\cellcolor[gray]{.9}&\cellcolor[gray]{.9}\\
%% \cline{3-6}
%  \cellcolor[gray]{.9}\multirow{-2}{*}{\scriptsize Protocols} &\cellcolor[gray]{.9}\multirow{-2}{*} {\scriptsize Operation}&\cellcolor[gray]{.9}\scriptsize$\mathtt{Store}$&\cellcolor[gray]{.9}\scriptsize$\mathtt{SolvPuz}$&\cellcolor[gray]{.9}\scriptsize$\mathtt{Prove}$&\cellcolor[gray]{.9}\scriptsize$\mathtt{Verify}$&\cellcolor[gray]{.9}\multirow{-2}{*} {\scriptsize I/O Cost}&\cellcolor[gray]{.9}\multirow{-2}{*} {\scriptsize Client}&\cellcolor[gray]{.9}\multirow{-2}{*} {\scriptsize Server}&\cellcolor[gray]{.9}\multirow{-2}{*} {\scriptsize Verifier}&\cellcolor[gray]{.9}\multirow{-2}{*} {\scriptsize Proof Size}
%  \\
%\hline
%    %SO-PoR 1st row
%\cellcolor[gray]{.9}& \multirow{2}{*}{\rotatebox[origin=c]{0}{\cellcolor[gray]{.9}\scriptsize }} \scriptsize Exp.&\scriptsize$z+1$&\scriptsize$T z$&\scriptsize$-$ &$-$&&\scriptsize $128(n+$&&&\\
%     \cline{2-6}  
%     %SO-PoR 2nd row
% \multirow{-2}{*}{\rotatebox[origin=c]{0}{\cellcolor[gray]{.9}\scriptsize  SO-PoR }}&\cellcolor[gray]{.9}\scriptsize Add. or Mul.&\scriptsize$2(n+cz)$ &\scriptsize$z$&\scriptsize$4 c z$&\scriptsize$2z(1+c)$&\multirow{-2}{*} {\scriptsize $O(1)$}&\scriptsize  $cz+19z)$&\multirow{-2}{*}{\scriptsize $884z$}&\multirow{-2}{*}{\scriptsize -}&\multirow{-2}{*}{\scriptsize $O(1)$}\\
%     \cline{2-6}   
%      
%     \hline 
%       %[3] 1st row 
%       
%          \hline 
%          
% \cellcolor[gray]{.9}  &\multirow{2}{*}{\rotatebox[origin=c]{0}{\scriptsize }}\cellcolor[gray]{.9}\scriptsize Exp.&\scriptsize$9 n$&\scriptsize$ -$&\scriptsize$-$&$-$&&&\scriptsize$256z+$&\scriptsize $4672n+$&\\
%     \cline{2-6}
%     %[3] 2nd row 
%\cellcolor[gray]{.9}   \multirow{-2}{*}{\rotatebox[origin=c]{0}{\scriptsize   \cite{armknecht2014outsourced}}}  &\cellcolor[gray]{.9}\scriptsize Add. or Mul.&\scriptsize$10 n$&$-$&\scriptsize$4 z(c+c')$&\scriptsize$z(9c+3)$&\multirow{-2}{*} {\scriptsize $O(1)$}&\multirow{-2}{*} {\scriptsize $128n$}&\scriptsize $||\vv{\bm{F}}||$&\scriptsize $256z$&\multirow{-2}{*} {\scriptsize $O(1)$}\\   
%      
%      \hline
%      
%       \hline
%      %[53] 1st row
%  \cellcolor[gray]{.9}    &\multirow{2}{*}{\rotatebox[origin=c]{0}{\  \scriptsize }}\cellcolor[gray]{.9}\scriptsize Exp.&\scriptsize$-$&$-$&\scriptsize$z (3+c)$&\scriptsize$6 z$&&&&&\\
%     \cline{2-6}
%     %[53] 2nd row
% \cellcolor[gray]{.9}&\cellcolor[gray]{.9}\scriptsize Add. or Mul.&\scriptsize$4 n$&$-$&\scriptsize$2 z(3 c+4)$&\scriptsize$2 c z$&&&&&\\ 
%    \cline{2-6}
%    %[53] 3rd row 
%\cellcolor[gray]{.9}\multirow{-3}{*}{\rotatebox[origin=c]{0}{\scriptsize   \cite{xu2016lightweight}}}&\cellcolor[gray]{.9}\scriptsize Pairing&\scriptsize$-$&\scriptsize$-$&\scriptsize$7 z$&$-$&\multirow{-3}{*} {\scriptsize $O(1)$}&\multirow{-3}{*} {\scriptsize $2048n$}&\multirow{-3}{*} {\scriptsize $6144z$}&\multirow{-3}{*} {-}&\multirow{-3}{*} {\scriptsize $O(1)$}\\ 
% \hline
%  
%  %%%%%%%%%%%%%%%%%%%%%%%%%%%%%%
% 
% \hline
%\cellcolor[gray]{.9}& \multirow{2}{*}{\rotatebox[origin=c]{0}{\cellcolor[gray]{.9}\scriptsize }} \scriptsize Exp.&\scriptsize$-$&\scriptsize$3T z$&\scriptsize$-$ &\scriptsize$3z$&&&\scriptsize$128cz \log n+$&&\\
%     \cline{2-6}  
% \multirow{-2}{*}{\rotatebox[origin=c]{0}{\cellcolor[gray]{.9}\scriptsize  \cite{Storage-Time} }}&\cellcolor[gray]{.9}\scriptsize Add. or Mul.&\scriptsize$-$ &\scriptsize$Tz$&\scriptsize$-$&\scriptsize$-$&\multirow{-2}{*} {\scriptsize $O(\log n)$}&\multirow{-2}{*} {\scriptsize $128$}&\scriptsize $4096z$&\multirow{-2}{*} {-}&\multirow{-2}{*} {\scriptsize$O(\log n)$} \\
%     \cline{2-6}    
%     \hline 
% 
% %%%%%%%%%%%%%%%%%%%%%%%%%%%%%%
% 
% 
% 
%\end{tabular}  %xxxxxx
%\end{minipage}
%\end{center}
%\end{footnotesize}
%\end{table*}
%

%--------------------------------------------------------




\noindent\textbf{\textit{Properties}}. Briefly, only SO-PoR offers an explicit solution for real-time detection. Unlike the rest, it also supports efficient fair payment. The authors of \cite{Storage-Time}  briefly suggest that in their  ``basic PoSt''  protocol the verification can be performed by a smart contract, so it can pay the server after approving all proofs, i.e. fair payment. However, this is  only of \emph{theoretical interest}, as in practice its verification algorithm imposes significant computation, communication and financial cost to the smart contract and the user (higher than the costs of strawman solutions discussed in Section \ref{Strawman-Solutions}). Furthermore,  a client in  \cite{xu2016lightweight} has to fully trust the auditor with the correctness of verification, but this assumption is relaxed in   SO-PoR and  \cite{armknecht2014outsourced}, and  if a contract performs the verification in \cite{Storage-Time}.  %Also, the protocol in \cite{armknecht2014outsourced} is the only outsourced PoR secure against a malicious client. 



\noindent\textbf{\textit{Computation Complexity}}.  The verification in SO-PoR is much faster than the other three protocols; firstly, it requires no exponentiations in this phase, whereas \cite{xu2016lightweight,Storage-Time} do require a high number of exponentiations (i.e. $6z$ and $3z$ respectively), and secondly, it requires $\frac{9c+3}{2(1+c)}$ times fewer computation than \cite{armknecht2014outsourced};  Specifically, when $c=460$,  SO-PoR verification requires about $4.5$ times fewer computation than the verification in \cite{armknecht2014outsourced} needs\footnote{As shown in \cite{DBLP:conf/ccs/AtenieseBCHKPS07}, to ensure $99\%$ of file blocks is retrievable, it  suffices to set $c=460$.}. In SO-PoR, the cloud server needs to perform $Tz$ exponentiations to solve puzzles, however this is independent of the file size. The sever in \cite{Storage-Time} also performs $Tz$ modular multiplications and $3Tz$ modular exponentiations, which is  $3$ times higher than the  exponentiations done by the server in SO-PoR. The  protocols in \cite{armknecht2014outsourced,DBLP:conf/asiacrypt/ShachamW08} do not include the puzzle-solving procedure (and they do not offer all features that SO-PoR does).  Furthermore, the proving cost in SO-PoR is similar to that of in \cite{armknecht2014outsourced}, and is  much better than \cite{xu2016lightweight}, as the latter one requires both exponentiations and pairing operations while the prove algorithm in SO-PoR  does not involve any exponentiations. The  proving cost in \cite{Storage-Time} is the lowest, as it requires only invocations of a hash function.  Also, the store phase in SO-PoR has a much lower computation cost than the one in \cite{armknecht2014outsourced}. The reason is that the number of exponentiations required (in this phase) in SO-PoR is independent of file size and is only linear with the number of delegated verifications; however, the number of exponentiations in \cite{armknecht2014outsourced} is linear with the file size. For instance, when $||\vv{\bm{F}}||=1$-GB, the  total number of blocks is   $n=\frac{1-\text{GB}}{128-\text{bit}}=625\times 10^{\scriptscriptstyle 5}$. Since the number of exponentiations in \cite{armknecht2014outsourced} is linear with the number of blocks, i.e. $9n$, the total number of exponentiations imposed by the store algorithm is $5625\times 10^{\scriptscriptstyle 5}$ which is very high. This is the reason why in the experiment in \cite{armknecht2014outsourced} only a small file size $64$-MB, is used, that can be stored locally without the need to use  cloud storage, in the first place.  Now, we turn our attention to SO-PoR. Let the verification be done every month for a $10$-year period, in this case,  $z=120$.  So, the total number of exponentiations required by the store in SO-PoR is $121$. This means store phase in SO-PoR requires over $46\times 10^{\scriptscriptstyle 5}$ times  fewer exponentiations than the one in \cite{armknecht2014outsourced} needs. The store algorithm in \cite{xu2016lightweight} does not involve any exponentiations; however, its  number of modular additions and multiplication  is higher than the ones imposed by the SO-PoR's store. The  store cost in \cite{Storage-Time} is the lowest, as it requires only invocations of a hash function. Furthermore, the verification and prove cost of SO-PoR and privately verifiable PoR \cite{DBLP:conf/asiacrypt/ShachamW08} are identical. Thus, SO-PoR offers a very efficient verification algorithm that allows it to employ a smart contract to perform the verification, while its other computation costs are also reasonably low. But, this is not the case for the other protocols. 

%The verification in SO-PoR is  faster than the other two protocols; firstly, it requires no exponentiations, whereas \cite{xu2016lightweight} does, and secondly, it requires $\frac{9c+3}{2(1+c)}$ times fewer computation than \cite{armknecht2014outsourced}.  Specifically, when $c=460$,  SO-PoR verification requires about $4.5$ times fewer computations than the verification in \cite{armknecht2014outsourced} needs\footnote{As shown in \cite{DBLP:conf/ccs/AtenieseBCHKPS07}, to ensure $99\%$ of file blocks is retrievable, it would suffice to set $c=460$.}. In SO-PoR, the cloud   performs $Tz$ exponentiations to solve puzzles,  independent of the file size. The other two protocols do not include  a puzzle-solving procedure (and they do not offer all features that SO-PoR does). The proving cost in SO-PoR is similar to that of in \cite{armknecht2014outsourced}, and is  much better than \cite{xu2016lightweight}, as the latter one requires additional  exponentiations and pairing operations. The store phase in SO-PoR has a much lower computation cost than the one in \cite{armknecht2014outsourced}.  For instance, when $||\vv{\bm{F}}||=1$-GB, the  total number of blocks is:   $n=\frac{1-\text{GB}}{128-\text{bit}}=625\times 10^{\scriptscriptstyle 5}$. Since the number of exponentiations in \cite{armknecht2014outsourced} is linear with the number of blocks, i.e. $9n$, the total number of exponentiations imposed by the store algorithm is: $5625\times 10^{\scriptscriptstyle 5}$ which is very high. Now, we turn our attention to SO-PoR. Let the verification be done every month for  $10$ years period, in this case,  $z=120$.  So, the total number of exponentiations required by the store in SO-PoR is $121$. This means the store phase in SO-PoR needs over $46\times 10^{\scriptscriptstyle 5}$ times  fewer exponentiations than  \cite{armknecht2014outsourced} needs. The store algorithm in \cite{xu2016lightweight} does not involve any exponentiations. However, its  number of modular additions and multiplications  is higher than the ones imposed by SO-PoR' store algorithm. Also, the verification and prove costs in SO-PoR and in the very efficient privately verifiable PoR \cite{DBLP:conf/asiacrypt/ShachamW08} are identical.






\noindent\textbf{\textit{I/O Complexity}}. In SO-PoR and \cite{xu2016lightweight,armknecht2014outsourced}, for a server to generate a PoR, it only needs to access a constant number of file blocks. Therefore, their total I/O cost is $O(1)$. But,  in \cite{Storage-Time}  the server has to access a \emph{logarithmic number of file blocks} to generate a PoR, so its I/O cost is much higher than the rest. In particular, its total I/O cost is $O(\log n)$. Note that I/O cost plays a crucial role in the scalability of the server and its ability to serve multiple clients/queries degrades when the I/O cost is significantly high. 



 \noindent\textbf{\textit{Communication Complexity}}. A verifier-side bandwidth of SO-PoR (and \cite{xu2016lightweight,Storage-Time}) is much lower than \cite{armknecht2014outsourced}. For instance, when $||\vv{\bm{F}}||=1$-GB and $z=100$, a verifier in SO-PoR requires $62\times 10^{\scriptscriptstyle 6}$  fewer bits than the one in \cite{armknecht2014outsourced} does. A client in SO-PoR has a higher bandwidth than it would have in the rest of the protocols. But, this cost is one-off, at the setup phase.  The server-side bandwidth of SO-PoR is the lowest;  for instance (for the same parameters above) a server in SO-PoR requires $9\times 10^{\scriptscriptstyle4}$,  $7$, and $1729$ times fewer bits  than those required in \cite{armknecht2014outsourced}, \cite{xu2016lightweight} and \cite{Storage-Time} respectively.   Moreover, \cite{Storage-Time} has the worst proof size complexity, which is logarithmic to the file size; while the proof size complexity of the rest of the schemes  is constant.  Hence, SO-PoR's server-side bandwidth is significantly lower than the rest  while having constant proof size. 
