% !TEX root =main.tex



\section{Smarter Outsourced PoR (SO-PoR) Utilising C-TLP}
   %\vspace{-5mm}
 As discussed in Section \ref{Related-Work}, the existing outsourced PoR's  have serious shortcomings, e.g. having high costs, not supporting real-time detection or suffering from the lack of a fair payment mechanism. In this section, we present SO-PoR to addresses them. 
 
 


 
\subsection{SO-PoR Overview} 

%SO-PoR uses a unique combination of (a) homomorphic MAC-based PoR \cite{DBLP:conf/asiacrypt/ShachamW08}, (b) C-TLP, and (c) a smart contract. It uses the MAC-based PoR, due to its high efficiency. Since the MAC's are privately verifiable and secret verification keys are needed to check PoR proofs, it also uses C-TLP to efficiently make them publicly verifiable. In this case, C-TLP encapsulates the verification keys and reveals each of them to verifiers only after a certain time. This combination allows the protocol to \emph{take advantage of MAC's efficiency in the setting where public verifiability is needed}. The combination of the two primitives has applications beyond PoR. SO-PoR also utilises a smart contract who acts as a public verifier on the client's behalf to verify proofs and pay an honest server. The MAC's and CTL combination also makes it possible to use a smart contract (which does not inherently support a private state) while keeping costs very low. 




SO-PoR uses a unique combination of (a) homomorphic MAC-based PoR \cite{DBLP:conf/asiacrypt/ShachamW08}, (b) C-TLP,  (c) a smart contract,  (d) a pre-computation technique, and (e) blockchain-based random extraction beacon \cite{DBLP:journals/iacr/AbadiCKZ19,armknecht2014outsourced}. It uses the MAC-based PoR, due to its high efficiency. Since the MAC's are privately verifiable and secret verification keys are needed to check PoR proofs, SO-PoR also uses C-TLP to efficiently make them publicly verifiable. In this case, C-TLP encapsulates the verification keys and reveals each of them to verifiers only after a certain time. SO-PoR also utilises a smart contract which acts as a public verifier on the client's behalf to verify proofs and pay an honest server. The pre-computation technique allows the client at setup to generate a constant number of \emph{disposable} homomorphic MAC's for each verification.  The combination of disposable homomorphic MAC's and C-TLP  makes it possible to (a) use a smart contract  and (b) take advantage of MAC's efficiency in the setting where public verifiability is needed. This combination has applications beyond PoR.  A blockchain-based random extraction beacon allows the server to independently  derive a set of unpredictable random values from the blockchain such that the values' correctness is publicly verifiable. 



At a high-level  SO-PoR works as follows. The client encodes its file using an error-correcting code and  for each $j\text{\small{-th}}$ verification it does the following. It picks two   random keys: $(v_{\scriptscriptstyle j},l_{\scriptscriptstyle j})$ of a $\mathtt{PRF}$. It uses $v_{\scriptscriptstyle j}$ to generate $c$ random blocks' indices, i.e. challenged blocks. It utilises  $l_{\scriptscriptstyle j}$ to generate a disposable MAC on each challenged block. It also uses C-TLP to make two puzzles, one that encapsulates $v_{\scriptscriptstyle j}$, and  another that encapsulates $l_{\scriptscriptstyle j}$. It deposits enough coins to cover $z$ successful PoR verifications in a smart contract. The client sends the encoded file, tags and the puzzles to the server. When $j\text{\small{-th}}$ PoR proof is needed, the server manages to discover key $v_{\scriptscriptstyle j}$ that lets it determine which file blocks are challenges. The server also uses the beacon to extract a set of random values from the blockchain. Using the MAC's,  challenged blocks, and  beacon's outputs, the server generates a compact PoR proof. The server sends the proof to the contract. After that, it can delete the related disposable MAC's.  For the same verification, after a fixed time, it manages to find the related MAC's verification key: $l_{\scriptscriptstyle j}$. It sends the key to the server who checks the correctness of $l_{\scriptscriptstyle j}$ and  PoR proof. If the contract accepts all proofs, then it pays the server  for $j\text{\small{-th}}$ verification; otherwise, it notifies the client.  







%\subsection {SO-PoR Model Overview}\label{SO-PoR-Model}
%SO-PoR model is built upon the traditional PoR paradigm \cite{DBLP:conf/asiacrypt/ShachamW08} which   is a challenge-response  protocol where a server proves to  an honest client that its file is retrievable (see  Appendix \ref{PoR-Model} for a formal definition of \cite{DBLP:conf/asiacrypt/ShachamW08}). In  SO-PoR, however,  a client may not be available for   verification. So, it wants to  delegate  a set of verifications that it cannot carry out. Informally, in this setting, it (in addition to file retrievability)  must have three guarantees: (a) \emph{verification correctness}: every verification is performed honestly, so  the client can trust the verification's result  without redoing it, (b) \emph{real-time detection}: the client is notified in almost real-time when a  proof is rejected, and (c) \emph{fair payment}: in every verification, the server is paid only if a  proof  is accepted. In SO-PoR, three parties are involved: a client (honest), potentially malicious server  and a standard smart contract. SO-PoR also, analogous to  \cite{DBLP:conf/asiacrypt/ShachamW08},  allows a client to perform the verification itself,  when it is available. We refer readers to  Appendix \ref{SO-PoR-Model} for SO-PoR formal definition.  
%
%


% !TEX root =main.tex




\section {SO-PoR Model}\label{SO-PoR-Model}
In this section, we provide a formal definition  of SO-PoR. As previously stated, it builds upon the traditional PoR model \cite{DBLP:conf/asiacrypt/ShachamW08}, presented in Appendix \ref{PoR-Model}.  In  SO-PoR, unlike the traditional PoR, a client may not be available every time  verification is needed. Therefore, it wants to  delegate  a set of verifications that it cannot carry out itself. In this setting, it (in addition to file retrievability)  must have three guarantees: (a) \emph{verification correctness}: every verification is performed honestly, so  the client can rely on the verification result  without the need to re-do it, (b) \emph{real-time detection}: the client is notified in almost real-time when server's  proof is rejected, and (c) \emph{fair payment}: in every verification, the server is paid only if the server's  proof  is accepted. In SO-PoR, three parties are involved: an honest client, potentially malicious server  and a standard smart contract. SO-PoR also allows a client to perform the verification itself, analogous to the traditional  PoR, when it is available. 

%To satisfy the aforementioned requirements, and keep verifications' cost low, SO-PoR mainly utilises a smart contract (for verification and payment) and the chained time-lock puzzle to eventually release secret values used to: (a) generate challenges and (b) verify proofs. Therefore, i


\begin{definition}
A Smart Outsourced PoR (SO-PoR) scheme consists of seven algorithms ($\mathtt{Setup}, \mathtt{Store},$ $ \mathtt {SolvPuz}, $ $ \mathtt{GenChall}, \mathtt{Prove},$ $ \mathtt{Verify},  \mathtt{Pay}$) defined below: 


\
\begin{itemize}
\item[$\bullet$] $\mathtt{Setup}(1^{\scriptscriptstyle\lambda},\Delta, z)\rightarrow (\hat{sk},\hat{pk})$:  a probabilistic algorithm, run by a client.  It  takes as input a security: $1^{\scriptscriptstyle\lambda}$, time parameter: $\Delta$, and the number of verification delegated: $z$. It  outputs a set of  secret and public keys.

\

\item[$\bullet$] $\mathtt{Store}(\hat{sk},\hat{pk}, F,z)\rightarrow ({\bm{F}}, \sigma, \vv{\bm{o}},aux)$: a probabilistic algorithm, run only once by a client. It  takes as  input the secret key: $\hat{sk}$, public key: $\hat{pk}$, a file: $F$, and the number of verifications: $z$ that the client wants to delegate. It outputs an encoded file: ${\bm{F}}$,  a set of tags: $\sigma$, a set of $z$ puzzles: $\vv{\bm{o}}$, and public auxiliary data: $aux$. First three outputs are stored on the server and last output: $aux$, is   stored on a smart contract. 

\

\item[$\bullet$] $\mathtt {SolvPuz}(\hat{pk},\vv{\bm{o}})\rightarrow \vv{\bm{s}}$:  a deterministic algorithm that takes as input the public key: $\hat{pk}$ and puzzle vector: $\vv{\bm{o}}$.  It for each  $j\text{\small{-th}}$ verification outputs a  pair: $\ddot{s}_{\scriptscriptstyle j}:(v_{\scriptscriptstyle j},l_{\scriptscriptstyle j})$ of solutions, where $v_{\scriptscriptstyle j}$ and $l_{\scriptscriptstyle j}$ are outputted at time $t_{\scriptscriptstyle j}$ and $t'_{\scriptscriptstyle j}$ respectively and $t'_{\scriptscriptstyle j}> t_{\scriptscriptstyle j}$. Therefore, the algorithm in total outputs $z$ pairs. Value $l_{\scriptscriptstyle j}$ is sent  to the smart contract right after it is discovered. This algorithm is run  by the server.

\


\item[$\bullet$] $\mathtt{GenChall}(j,|{\bm{F}}|, 1^{\scriptscriptstyle\lambda},\ddot{s}_{\scriptscriptstyle j},aux)\rightarrow \vv{\bm{c}}$: a probabilistic algorithm that takes as input a verification index: $j$, the encoded file size: $|{\bm{F}}|$, security parameter: $1^{\scriptscriptstyle\lambda}$, first component of the related solution pair, $v_{\scriptscriptstyle j}\in \ddot{s}_{\scriptscriptstyle j}$, and public parameters: $pp\in aux$ containing  a blockchain and its parameters. It outputs pairs $\ddot{c}_{\scriptscriptstyle j} : (x_{\scriptscriptstyle j} , y_{\scriptscriptstyle j} )$, where each pair includes a pseudorandom  block's index:  $x_{\scriptscriptstyle j}$ and random coefficient: $y_{\scriptscriptstyle j}$. Also, values $x_{\scriptscriptstyle j}$ are derived from $v_{\scriptscriptstyle j}$ while $y_{\scriptscriptstyle j}$ are derived from $pp$. This algorithm is run by the server for each verification. 


%$pp$ is a public parameters for the beacon and it includes, blockchain, chain quality, and index. $\mathtt{GenCoeffs}()$ is called here

\

\item[$\bullet$] $\mathtt{Prove}(j,{\bm{F}}, \sigma,  \vv{\bm{c}})\rightarrow \pi$: a probabilistic algorithm that takes the verification index $j$, encoded file: ${\bm{F}}$ , (a subset of) tags: $\sigma$, and a vector of unpredictable challenges: $\vv{\bm{c}}$, as inputs and outputs a proof of  file retrievability. It is run by the server for each verification.

\

\item[$\bullet$] $\mathtt{Verify}(j,\pi,\ddot{s}_{\scriptscriptstyle j},aux)\rightarrow d:\{0,1\}$: a deterministic algorithm that takes the verification index $j$, proof: $\pi$,  second component of the related solution pair: $l_{\scriptscriptstyle j}\in \ddot{s}_{\scriptscriptstyle j}$, and public auxiliary data: $aux$.  If the proof is accepted, it outputs $d=1$; otherwise, outputs $d=0$. The default value of $d$ is $0$. This algorithm is run by the smart contract for each verification and invoked only once for each verification by only the server. 

\

\item[$\bullet$] $\mathtt{Pay}(j,d)\rightarrow d'=\{0,1\}$: a deterministic algorithm that takes the verification index $j$, the verification output: $d$. If $d=1$, it transfers $e$ amounts to the server and outputs $1$. Otherwise, it does not transfer anything, and outputs $0$. The default value of $d'$ is $0$. The algorithm is run by the  contract, and  invoked only by $\mathtt{Verify}(.)$. 
\end{itemize}
\end{definition}





%
%note that in the above, $\mathtt{Store}$ is a wrapper function that calls $\mathtt{GenPuz}(\vv{\bm{m}},\hat{sk},\hat{pk})$ and $\mathtt{Store}(\hat{sk},F)$ as subroutine,{\color{blue}xx explain what $\vv{\bm{m}}$ is for}
%




  
A SO-PoR scheme must satisfy two main properties: \emph{correctness} and \emph{soundness}. The correctness requires, for any: file, public-private key pairs, and puzzle solutions, both the verification  and pay algorithms, i.e. $\mathtt{Verify}(.)$ and $\mathtt{Pay}(.)$, output $1$ when interacting with  the  prover, verifier, and client  all of which are honest.  The soundness however is split into four properties: extractability, verification correctness, real-time detection, and fair payment, formally defined below.  Before we define the first property,  extractability, we provide the following  experiment between an environment: $\mathcal{E}$ and  adversary: $\mathcal{A}$ who corrupts $C\subsetneq\{\mathcal{S},\mathcal{M}_{\scriptscriptstyle 1},...,\mathcal{M}_{\scriptscriptstyle\beta}\}$, where $\beta$ is the maximum number of miners which can be corrupted in a secure blockchain. In this game, $\mathcal{A}$ plays the role of corrupt parties and $\mathcal{E}$ simulating an honest party's role. 


\begin{enumerate}
\item $\mathcal{E}$ executes $\mathtt{Setup}(.)$ algorithm and provides public key: $\hat{pk}$, to $\mathcal{A}$.   
\item $\mathcal{A}$ can pick  arbitrary file $F'$, and  uses it to make queries to  $\mathcal{E}$ to run:  $\mathtt{Store}(\hat{sk},\hat{pk},$ $ F',z)$ $\rightarrow (F'^{\scriptscriptstyle *}, \sigma, \vv{\bm{o}},aux)$  and return the output to $\mathcal{A}$. Also, upon receiving the output of $\mathtt{Store}()$, $\mathcal{A}$ can locally run  algorithms: $\mathtt {SolvPuz}(\hat{pk},\vv{\bm{o}})$ and   $\mathtt{GenChall}(j,$ $|F'^{\scriptscriptstyle *}|, $ $ 1^{\scriptscriptstyle\lambda},\ddot{s}_{\scriptscriptstyle j},aux)\rightarrow \vv{\bm{c}}$ as well as  $\mathtt{Prove}(j,F^{\scriptscriptstyle *}, \sigma, $ $ \vv{\bm{c}})\rightarrow \pi$,  to get their outputs as well. 
\item $\mathcal{A}$ can request $\mathcal{E}$ the execution of $\mathtt{Verify}(j,\pi,\ddot{s}_{\scriptscriptstyle j},aux)$ for any $F'$ used to query $\mathtt{Store}()$. Accordingly, $\mathcal{E}$ informs  $\mathcal{A}$ about the verification output. The adversary can send a polynomial number of queries to $\mathcal{E}$. Finally, $\mathcal{A}$ outputs the description of a prover: $\mathcal{A}'$ for any file it has already chosen above. 
\end{enumerate}

It is said a cheating prover: $\mathcal{A}'$ is $\epsilon$-admissible if it convincingly answers $\epsilon$ fraction of verification challenges \cite{DBLP:conf/asiacrypt/ShachamW08}. Informally, a SO-PoR scheme supports extractability, if there is an extractor algorithm: $\mathtt{Ext}(\hat{sk},\hat{pk},\mathtt{P}')$, that takes the secret-public keys and the description of the  machine implementing the prover's role: $\mathcal{A}'$ and outputs the file: $F'$. The extractor can reset the adversary to the beginning of the challenge phase and repeat this step polynomially many times for  of extraction, i.e. the extractor can rewind it.

\begin{definition}[$\epsilon$-extractable]\label{extractable} A SO-PoR scheme is $\epsilon$-extractable if  for every adversary: $\mathcal{A}$ who corrupts $C\subsetneq\{\mathcal{S},\mathcal{M}_{\scriptscriptstyle 1} $ $,..., \mathcal{M}_{\scriptscriptstyle\beta}\}$, plays the experiment above, and outputs an $\epsilon$-admissible cheating prover: $\mathcal{A}'$ for a file $F'$,  there exists an extraction algorithm that recovers $F'$ from $\mathcal{A}'$, given honest parties public-private keys and $\mathcal{A}'$,  i.e. $\mathtt{Ext}(\hat{sk},\hat{pk},\mathcal{A}')\rightarrow F'$, except with a negligible probability. 
\end{definition}

% . The extractor has the ability to reset the adversary to the beginning of the challenge phase and repeat this step polynomially many times for the purpose of extraction

In the above game, the environment, acting on honest parties' behalf, performs the verification correctly; which is not always the case in SO-PoR. As the verification can be run by miners a subset of which are potentially corrupted. Even in this case, the verification correctness must hold, e.g.  if a corrupt server sends an  invalid proof then even if $\beta-1$ miners are corrupt (and colluding with it) the verification function will not output $1$ and if the server is honest and submits a valid proof then the verification function does not output $0$ even if $\beta$ miners are corrupt, except with a negligible probability. This is formalised below. 


\begin{definition}[Verification Correctness]\label{Verification-Correctness} Let $\beta$ be the maximum number of miners that can be corrupted in a secure blockchain network and $\lambda'$ be the blockchain security parameter. Also, let $\mathcal{A}$ be the adversary who (plays the above game and) corrupts parties in either $C\subseteq\{\mathcal{S},\mathcal{M}_{\scriptscriptstyle 1},...,\mathcal{M}_{\scriptscriptstyle\beta-1}\}$ or $C'\subseteq\{\mathcal{M}_{\scriptscriptstyle 1},...,\mathcal{M}_{\scriptscriptstyle\beta}\}$.  In SO-PoR, we say the correctness of $j\text{\small-th}$ verification  is guaranteed if: 
 
$$\begin{array}{l}
\text{in the former case}: Pr[\mathtt{Verify}_{\scriptscriptstyle C}(j,\pi,\ddot{s}_{\scriptscriptstyle j},aux)=1]\leq \mu(\lambda')\\
\text{in the latter case}: Pr[\mathtt{Verify}_{\scriptscriptstyle C'}(j,\pi,\ddot{s}_{\scriptscriptstyle j},aux)=0]\leq \mu(\lambda')
\end{array}$$
where $\mu(.)$ is a negligible function. 
\end{definition}

Also, a client needs to have a guarantee that for each verification it can get a correct result within a (fixed) time period. 

\begin{definition}[$\Upsilon$-real-time Detection]\label{real-time Detection} Let $\mathcal{A}$, as defined above, be the adversary who corrupts either $C$ or $C'$.
A client, for each $j\text{\small{-th}}$ delegated verification, will get a correct output of  $\mathtt{Verify(.)}$, by  means of reading a blockchain, within time window $\Upsilon$, after the time when the server is supposed to send  its proof  to the blockchain network. Formally,

$$\mathtt{Read}(\Upsilon,\mathtt{Verify}_{\scriptscriptstyle D}(j,\pi,\ddot{s}_{\scriptscriptstyle j},aux))\rightarrow \{0,1\}$$
where $D\subsetneq\{C,C'\}$, except with a negligible probability. 
\end{definition}





\begin{definition}[Fair Payment]\label{Fair-Payment}  SO-PoR supports a fair payment if the client and server fairness are satisfied: 

\begin{itemize}
\item[$\bullet$] \textit{\textbf{Client Fairness}}: An honest client is guaranteed that it only pays ($e$ coins) if the server provides an accepting proof, except with a negligible probability. 
\item[$\bullet$]\textit{\textbf{Server Fairness}}: An honest server is guaranteed that the client gets a correct proof if the client pays ($e$ coins),   except with a negligible probability. 
\end{itemize}
Formally, let $\mathcal{A}$ be the adversary who corrupts either $C$ or $C'$, as defined above. To satisfy a fair payment:
\begin{equation}
Pr[\mathtt{Pay}_{\scriptscriptstyle D}(.)=b 	\cap  \mathtt{Verify}_{\scriptscriptstyle D}(.)=b]\geq 1-\mu(\lambda'),   
\end{equation}

the following inequality must hold:
\begin{equation}\label{inequ::fair-payment}
Pr[\mathtt{Pay}_{\scriptscriptstyle D}(.)=b' 	\cap \mathtt{Verify}_{\scriptscriptstyle D}(.)=b] \leq \mu(\lambda'),
\end{equation}
where $D\subsetneq\{C,C'\},b\neq b'$, and $b, b'\subsetneq\{0,1\}$









%\begin{equation}
%Pr[\mathtt{Pay}_{\scriptscriptstyle D}(.)=1 	\cap \mathtt{Verify}_{\scriptscriptstyle D}(.)=0] \leq \mu(\lambda')
%\end{equation}
%\begin{equation}
%Pr[ \mathtt{Pay}_{\scriptscriptstyle D}(.)=0 	\cap \mathtt{Verify}_{\scriptscriptstyle D}(.)=1] \leq \mu(\lambda')
%\end{equation}
%\begin{equation}
%Pr[\mathtt{Pay}_{\scriptscriptstyle D}(.)=1 	\cap  \mathtt{Verify}_{\scriptscriptstyle D}(.)=1]\geq 1-\mu(\lambda')
%\end{equation}
%where $D\subsetneq\{C,C'\}$
\end{definition}

The above definition also takes into account the fact that the client at the time of delegated verification is not necessarily available to make the payment itself, so the payment is delegated to a third party, e.g. a smart contract. In this case, the definition  ensures that even if  the client or/and server are honest, the third party cannot affect  the fairness (except with a negligible probability).

% Moreover, it is not hard to see, if the inequality \ref{} holds, then the fairness is guaranteed, with a high probability:  \begin{equation*}
%Pr[\mathtt{Pay}_{\scriptscriptstyle D}(.)=b 	\cap  \mathtt{Verify}_{\scriptscriptstyle D}(.)=b]\geq 1-\mu(\lambda')
%\end{equation*} 


%In the following we explain the rational behind the above definition. In SO-PoR scheme, for each verification, the server sells  a proof: $\pi$ to a client and earns  $e$ coins if and only if the proof is accepted, i.e. $\mathtt{Verify}(.)=1$. In SO-PoR setting, the client at the time of delegated verification is not necessarily online to make the payment itself, so it is done by a third party (e.g. a smart contract). The definition must ensure that   even if both server and client are honest, the third party cannot affect  the fairness. 


\begin{definition}[SO-PoR Security]\label{SO-PoR-Security} A SO-PoR scheme is secure if it is $\epsilon$-extractable, and satisfies verification correctness, $\Upsilon$-real-time detection, and fair payment properties.

\end{definition}




\begin{remark}
The folklore assumption is that (in a secure blockchain) a smart contract function \emph{always outputs a correct result}. However, this is not the case and it may fail under certain circumstances.  For instance, as shown in \cite{LuuTKS15} all rational  miners may not verify a certain transaction. As another example,  an adversary (although with a negligibly small probability)  discards a  certain honestly generated blocks,  reverses the state of blockchain and contract, or breaks a client's signature scheme.  Accordingly, in our definitions above, we take such cases  into consideration and allow the possibility that a function outputs an incorrect result even though with a negligibly small probability. 
\end{remark}




\begin{remark}
SO-PoR model differs from traditional (e.g. \cite{DBLP:conf/ccs/JuelsK07,DBLP:conf/asiacrypt/ShachamW08}) and outsourced PoR  (e.g. \cite{armknecht2014outsourced,xu2016lightweight}) models in several aspects. Only  the SO-PoR model offers all the properties. In particular,  traditional PoRs only offer extractability while outsourced ones  offer liability as well, that allows a client (by re-running all verifications function) to detect a verifier if it provides an incorrect verification output, so the client   cannot rely on the verification result provided.  As another difference, the SO-PoR model  takes into account the case where an adversary can corrupt both the server and some miners at the same time.
\end{remark}

\begin{remark}
SO-PoR should also support  the traditional PoR where only client and server interact with each other  (e.g. client generates challenges, and verifies proof) when the client is available. To let SO-PoR definition support that too, we can simply define a flag: $\xi$, in each function, such that  when $\xi=1$, it acts as the traditional PoR; otherwise (when $\xi=0$), it performs as a delegated one. For the sake of simplicity, we let the flag  be implicit in the definitions above, where the default  is  $\xi=1$ 
\end{remark}


%SO-PoR uses a novel combination of disposable tags, pre-computation technique,  C-TLP scheme, smart contract,  pseudorandom functions and commitment scheme.  At a high-level, SO-PoR works as follows. The client generates a set of authenticator tags on the file. These tags will allow the client to verify its data availability  when it is online. Also, for every verification that the client cannot be online, it precomputes a (small) set of \emph{disposable} tags related to the file's blocks that will be challenged for that verification. However, unlike standard PoR schemes in which   a subset of file blocks are challenged by picking their indices  randomly on the fly just before the verification, in SO-PoR, the challenged blocks' indices are picked \emph{pseudorandomly} by the client in the setup phase. Then, the client for, each verification, encodes the secret key that allows regeneration of the pseudorandom indices and a secret key used for a PoR verification into two puzzles. The client stores the file, tags, and puzzles on the server. It stores  commitments of the secret values that will be used for PoR verification  in a smart contract. Also, it deposits enough coins in the smart contract, to pay the server if each proof (given by the server) is accepted. At each verification time, the server first solves a puzzle and fully recovers the key for random indices. Using the key,  corresponding tags and the file, it generates a PoR and sends it to the contract. After a certain  period, for the same verification, it manages to fully find another puzzle solution  which is the verification key. It sends the key to the contract  who first checks the correctness of the key and then verifies the PoR. If accepted, the contract  for that verification pays the cloud server  who can now delete all metadata (e.g. tags, encrypted, and decrypted values) for that verification.  



%First, client breaks up its file into blocks and apply an error-correcting code on every blocks. Then, it generates a set of MAC-based tags on every  blocks. These tags will allow the client to verify its data availability when it is online. For the sake of simplicity, let us assume the client does not want to perform  $z$ consecutive verifications. For  every $j^{\scriptscriptstyle th}$ verification  ($1\leq j\leq z$)  the client determines the random indices of the blocks that will be challenged for this verification and also precomputes (small) set of MAC-based tags for those blocks. It uses time-lock encryption scheme to encrypts the random indices and secret verification values (for the tags) and stores the encrypted values on the server. It also  stores the hash of random indices and  secret verification values in a smart contract. The client also stores the indices of the blocks that will appear in the blockchain from which a set of random value will be extracted. 

%% !TEX root =main.tex




\section {SO-PoR Model}\label{SO-PoR-Model}
In this section, we provide a formal definition  of SO-PoR. As previously stated, it builds upon the traditional PoR model \cite{DBLP:conf/asiacrypt/ShachamW08}, presented in Appendix \ref{PoR-Model}.  In  SO-PoR, unlike the traditional PoR, a client may not be available every time  verification is needed. Therefore, it wants to  delegate  a set of verifications that it cannot carry out itself. In this setting, it (in addition to file retrievability)  must have three guarantees: (a) \emph{verification correctness}: every verification is performed honestly, so  the client can rely on the verification result  without the need to re-do it, (b) \emph{real-time detection}: the client is notified in almost real-time when server's  proof is rejected, and (c) \emph{fair payment}: in every verification, the server is paid only if the server's  proof  is accepted. In SO-PoR, three parties are involved: an honest client, potentially malicious server  and a standard smart contract. SO-PoR also allows a client to perform the verification itself, analogous to the traditional  PoR, when it is available. 

%To satisfy the aforementioned requirements, and keep verifications' cost low, SO-PoR mainly utilises a smart contract (for verification and payment) and the chained time-lock puzzle to eventually release secret values used to: (a) generate challenges and (b) verify proofs. Therefore, i


\begin{definition}
A Smart Outsourced PoR (SO-PoR) scheme consists of seven algorithms ($\mathtt{Setup}, \mathtt{Store},$ $ \mathtt {SolvPuz}, $ $ \mathtt{GenChall}, \mathtt{Prove},$ $ \mathtt{Verify},  \mathtt{Pay}$) defined below: 


\
\begin{itemize}
\item[$\bullet$] $\mathtt{Setup}(1^{\scriptscriptstyle\lambda},\Delta, z)\rightarrow (\hat{sk},\hat{pk})$:  a probabilistic algorithm, run by a client.  It  takes as input a security: $1^{\scriptscriptstyle\lambda}$, time parameter: $\Delta$, and the number of verification delegated: $z$. It  outputs a set of  secret and public keys.

\

\item[$\bullet$] $\mathtt{Store}(\hat{sk},\hat{pk}, F,z)\rightarrow ({\bm{F}}, \sigma, \vv{\bm{o}},aux)$: a probabilistic algorithm, run only once by a client. It  takes as  input the secret key: $\hat{sk}$, public key: $\hat{pk}$, a file: $F$, and the number of verifications: $z$ that the client wants to delegate. It outputs an encoded file: ${\bm{F}}$,  a set of tags: $\sigma$, a set of $z$ puzzles: $\vv{\bm{o}}$, and public auxiliary data: $aux$. First three outputs are stored on the server and last output: $aux$, is   stored on a smart contract. 

\

\item[$\bullet$] $\mathtt {SolvPuz}(\hat{pk},\vv{\bm{o}})\rightarrow \vv{\bm{s}}$:  a deterministic algorithm that takes as input the public key: $\hat{pk}$ and puzzle vector: $\vv{\bm{o}}$.  It for each  $j\text{\small{-th}}$ verification outputs a  pair: $\ddot{s}_{\scriptscriptstyle j}:(v_{\scriptscriptstyle j},l_{\scriptscriptstyle j})$ of solutions, where $v_{\scriptscriptstyle j}$ and $l_{\scriptscriptstyle j}$ are outputted at time $t_{\scriptscriptstyle j}$ and $t'_{\scriptscriptstyle j}$ respectively and $t'_{\scriptscriptstyle j}> t_{\scriptscriptstyle j}$. Therefore, the algorithm in total outputs $z$ pairs. Value $l_{\scriptscriptstyle j}$ is sent  to the smart contract right after it is discovered. This algorithm is run  by the server.

\


\item[$\bullet$] $\mathtt{GenChall}(j,|{\bm{F}}|, 1^{\scriptscriptstyle\lambda},\ddot{s}_{\scriptscriptstyle j},aux)\rightarrow \vv{\bm{c}}$: a probabilistic algorithm that takes as input a verification index: $j$, the encoded file size: $|{\bm{F}}|$, security parameter: $1^{\scriptscriptstyle\lambda}$, first component of the related solution pair, $v_{\scriptscriptstyle j}\in \ddot{s}_{\scriptscriptstyle j}$, and public parameters: $pp\in aux$ containing  a blockchain and its parameters. It outputs pairs $\ddot{c}_{\scriptscriptstyle j} : (x_{\scriptscriptstyle j} , y_{\scriptscriptstyle j} )$, where each pair includes a pseudorandom  block's index:  $x_{\scriptscriptstyle j}$ and random coefficient: $y_{\scriptscriptstyle j}$. Also, values $x_{\scriptscriptstyle j}$ are derived from $v_{\scriptscriptstyle j}$ while $y_{\scriptscriptstyle j}$ are derived from $pp$. This algorithm is run by the server for each verification. 


%$pp$ is a public parameters for the beacon and it includes, blockchain, chain quality, and index. $\mathtt{GenCoeffs}()$ is called here

\

\item[$\bullet$] $\mathtt{Prove}(j,{\bm{F}}, \sigma,  \vv{\bm{c}})\rightarrow \pi$: a probabilistic algorithm that takes the verification index $j$, encoded file: ${\bm{F}}$ , (a subset of) tags: $\sigma$, and a vector of unpredictable challenges: $\vv{\bm{c}}$, as inputs and outputs a proof of  file retrievability. It is run by the server for each verification.

\

\item[$\bullet$] $\mathtt{Verify}(j,\pi,\ddot{s}_{\scriptscriptstyle j},aux)\rightarrow d:\{0,1\}$: a deterministic algorithm that takes the verification index $j$, proof: $\pi$,  second component of the related solution pair: $l_{\scriptscriptstyle j}\in \ddot{s}_{\scriptscriptstyle j}$, and public auxiliary data: $aux$.  If the proof is accepted, it outputs $d=1$; otherwise, outputs $d=0$. The default value of $d$ is $0$. This algorithm is run by the smart contract for each verification and invoked only once for each verification by only the server. 

\

\item[$\bullet$] $\mathtt{Pay}(j,d)\rightarrow d'=\{0,1\}$: a deterministic algorithm that takes the verification index $j$, the verification output: $d$. If $d=1$, it transfers $e$ amounts to the server and outputs $1$. Otherwise, it does not transfer anything, and outputs $0$. The default value of $d'$ is $0$. The algorithm is run by the  contract, and  invoked only by $\mathtt{Verify}(.)$. 
\end{itemize}
\end{definition}





%
%note that in the above, $\mathtt{Store}$ is a wrapper function that calls $\mathtt{GenPuz}(\vv{\bm{m}},\hat{sk},\hat{pk})$ and $\mathtt{Store}(\hat{sk},F)$ as subroutine,{\color{blue}xx explain what $\vv{\bm{m}}$ is for}
%




  
A SO-PoR scheme must satisfy two main properties: \emph{correctness} and \emph{soundness}. The correctness requires, for any: file, public-private key pairs, and puzzle solutions, both the verification  and pay algorithms, i.e. $\mathtt{Verify}(.)$ and $\mathtt{Pay}(.)$, output $1$ when interacting with  the  prover, verifier, and client  all of which are honest.  The soundness however is split into four properties: extractability, verification correctness, real-time detection, and fair payment, formally defined below.  Before we define the first property,  extractability, we provide the following  experiment between an environment: $\mathcal{E}$ and  adversary: $\mathcal{A}$ who corrupts $C\subsetneq\{\mathcal{S},\mathcal{M}_{\scriptscriptstyle 1},...,\mathcal{M}_{\scriptscriptstyle\beta}\}$, where $\beta$ is the maximum number of miners which can be corrupted in a secure blockchain. In this game, $\mathcal{A}$ plays the role of corrupt parties and $\mathcal{E}$ simulating an honest party's role. 


\begin{enumerate}
\item $\mathcal{E}$ executes $\mathtt{Setup}(.)$ algorithm and provides public key: $\hat{pk}$, to $\mathcal{A}$.   
\item $\mathcal{A}$ can pick  arbitrary file $F'$, and  uses it to make queries to  $\mathcal{E}$ to run:  $\mathtt{Store}(\hat{sk},\hat{pk},$ $ F',z)$ $\rightarrow (F'^{\scriptscriptstyle *}, \sigma, \vv{\bm{o}},aux)$  and return the output to $\mathcal{A}$. Also, upon receiving the output of $\mathtt{Store}()$, $\mathcal{A}$ can locally run  algorithms: $\mathtt {SolvPuz}(\hat{pk},\vv{\bm{o}})$ and   $\mathtt{GenChall}(j,$ $|F'^{\scriptscriptstyle *}|, $ $ 1^{\scriptscriptstyle\lambda},\ddot{s}_{\scriptscriptstyle j},aux)\rightarrow \vv{\bm{c}}$ as well as  $\mathtt{Prove}(j,F^{\scriptscriptstyle *}, \sigma, $ $ \vv{\bm{c}})\rightarrow \pi$,  to get their outputs as well. 
\item $\mathcal{A}$ can request $\mathcal{E}$ the execution of $\mathtt{Verify}(j,\pi,\ddot{s}_{\scriptscriptstyle j},aux)$ for any $F'$ used to query $\mathtt{Store}()$. Accordingly, $\mathcal{E}$ informs  $\mathcal{A}$ about the verification output. The adversary can send a polynomial number of queries to $\mathcal{E}$. Finally, $\mathcal{A}$ outputs the description of a prover: $\mathcal{A}'$ for any file it has already chosen above. 
\end{enumerate}

It is said a cheating prover: $\mathcal{A}'$ is $\epsilon$-admissible if it convincingly answers $\epsilon$ fraction of verification challenges \cite{DBLP:conf/asiacrypt/ShachamW08}. Informally, a SO-PoR scheme supports extractability, if there is an extractor algorithm: $\mathtt{Ext}(\hat{sk},\hat{pk},\mathtt{P}')$, that takes the secret-public keys and the description of the  machine implementing the prover's role: $\mathcal{A}'$ and outputs the file: $F'$. The extractor can reset the adversary to the beginning of the challenge phase and repeat this step polynomially many times for  of extraction, i.e. the extractor can rewind it.

\begin{definition}[$\epsilon$-extractable]\label{extractable} A SO-PoR scheme is $\epsilon$-extractable if  for every adversary: $\mathcal{A}$ who corrupts $C\subsetneq\{\mathcal{S},\mathcal{M}_{\scriptscriptstyle 1} $ $,..., \mathcal{M}_{\scriptscriptstyle\beta}\}$, plays the experiment above, and outputs an $\epsilon$-admissible cheating prover: $\mathcal{A}'$ for a file $F'$,  there exists an extraction algorithm that recovers $F'$ from $\mathcal{A}'$, given honest parties public-private keys and $\mathcal{A}'$,  i.e. $\mathtt{Ext}(\hat{sk},\hat{pk},\mathcal{A}')\rightarrow F'$, except with a negligible probability. 
\end{definition}

% . The extractor has the ability to reset the adversary to the beginning of the challenge phase and repeat this step polynomially many times for the purpose of extraction

In the above game, the environment, acting on honest parties' behalf, performs the verification correctly; which is not always the case in SO-PoR. As the verification can be run by miners a subset of which are potentially corrupted. Even in this case, the verification correctness must hold, e.g.  if a corrupt server sends an  invalid proof then even if $\beta-1$ miners are corrupt (and colluding with it) the verification function will not output $1$ and if the server is honest and submits a valid proof then the verification function does not output $0$ even if $\beta$ miners are corrupt, except with a negligible probability. This is formalised below. 


\begin{definition}[Verification Correctness]\label{Verification-Correctness} Let $\beta$ be the maximum number of miners that can be corrupted in a secure blockchain network and $\lambda'$ be the blockchain security parameter. Also, let $\mathcal{A}$ be the adversary who (plays the above game and) corrupts parties in either $C\subseteq\{\mathcal{S},\mathcal{M}_{\scriptscriptstyle 1},...,\mathcal{M}_{\scriptscriptstyle\beta-1}\}$ or $C'\subseteq\{\mathcal{M}_{\scriptscriptstyle 1},...,\mathcal{M}_{\scriptscriptstyle\beta}\}$.  In SO-PoR, we say the correctness of $j\text{\small-th}$ verification  is guaranteed if: 
 
$$\begin{array}{l}
\text{in the former case}: Pr[\mathtt{Verify}_{\scriptscriptstyle C}(j,\pi,\ddot{s}_{\scriptscriptstyle j},aux)=1]\leq \mu(\lambda')\\
\text{in the latter case}: Pr[\mathtt{Verify}_{\scriptscriptstyle C'}(j,\pi,\ddot{s}_{\scriptscriptstyle j},aux)=0]\leq \mu(\lambda')
\end{array}$$
where $\mu(.)$ is a negligible function. 
\end{definition}

Also, a client needs to have a guarantee that for each verification it can get a correct result within a (fixed) time period. 

\begin{definition}[$\Upsilon$-real-time Detection]\label{real-time Detection} Let $\mathcal{A}$, as defined above, be the adversary who corrupts either $C$ or $C'$.
A client, for each $j\text{\small{-th}}$ delegated verification, will get a correct output of  $\mathtt{Verify(.)}$, by  means of reading a blockchain, within time window $\Upsilon$, after the time when the server is supposed to send  its proof  to the blockchain network. Formally,

$$\mathtt{Read}(\Upsilon,\mathtt{Verify}_{\scriptscriptstyle D}(j,\pi,\ddot{s}_{\scriptscriptstyle j},aux))\rightarrow \{0,1\}$$
where $D\subsetneq\{C,C'\}$, except with a negligible probability. 
\end{definition}





\begin{definition}[Fair Payment]\label{Fair-Payment}  SO-PoR supports a fair payment if the client and server fairness are satisfied: 

\begin{itemize}
\item[$\bullet$] \textit{\textbf{Client Fairness}}: An honest client is guaranteed that it only pays ($e$ coins) if the server provides an accepting proof, except with a negligible probability. 
\item[$\bullet$]\textit{\textbf{Server Fairness}}: An honest server is guaranteed that the client gets a correct proof if the client pays ($e$ coins),   except with a negligible probability. 
\end{itemize}
Formally, let $\mathcal{A}$ be the adversary who corrupts either $C$ or $C'$, as defined above. To satisfy a fair payment:
\begin{equation}
Pr[\mathtt{Pay}_{\scriptscriptstyle D}(.)=b 	\cap  \mathtt{Verify}_{\scriptscriptstyle D}(.)=b]\geq 1-\mu(\lambda'),   
\end{equation}

the following inequality must hold:
\begin{equation}\label{inequ::fair-payment}
Pr[\mathtt{Pay}_{\scriptscriptstyle D}(.)=b' 	\cap \mathtt{Verify}_{\scriptscriptstyle D}(.)=b] \leq \mu(\lambda'),
\end{equation}
where $D\subsetneq\{C,C'\},b\neq b'$, and $b, b'\subsetneq\{0,1\}$









%\begin{equation}
%Pr[\mathtt{Pay}_{\scriptscriptstyle D}(.)=1 	\cap \mathtt{Verify}_{\scriptscriptstyle D}(.)=0] \leq \mu(\lambda')
%\end{equation}
%\begin{equation}
%Pr[ \mathtt{Pay}_{\scriptscriptstyle D}(.)=0 	\cap \mathtt{Verify}_{\scriptscriptstyle D}(.)=1] \leq \mu(\lambda')
%\end{equation}
%\begin{equation}
%Pr[\mathtt{Pay}_{\scriptscriptstyle D}(.)=1 	\cap  \mathtt{Verify}_{\scriptscriptstyle D}(.)=1]\geq 1-\mu(\lambda')
%\end{equation}
%where $D\subsetneq\{C,C'\}$
\end{definition}

The above definition also takes into account the fact that the client at the time of delegated verification is not necessarily available to make the payment itself, so the payment is delegated to a third party, e.g. a smart contract. In this case, the definition  ensures that even if  the client or/and server are honest, the third party cannot affect  the fairness (except with a negligible probability).

% Moreover, it is not hard to see, if the inequality \ref{} holds, then the fairness is guaranteed, with a high probability:  \begin{equation*}
%Pr[\mathtt{Pay}_{\scriptscriptstyle D}(.)=b 	\cap  \mathtt{Verify}_{\scriptscriptstyle D}(.)=b]\geq 1-\mu(\lambda')
%\end{equation*} 


%In the following we explain the rational behind the above definition. In SO-PoR scheme, for each verification, the server sells  a proof: $\pi$ to a client and earns  $e$ coins if and only if the proof is accepted, i.e. $\mathtt{Verify}(.)=1$. In SO-PoR setting, the client at the time of delegated verification is not necessarily online to make the payment itself, so it is done by a third party (e.g. a smart contract). The definition must ensure that   even if both server and client are honest, the third party cannot affect  the fairness. 


\begin{definition}[SO-PoR Security]\label{SO-PoR-Security} A SO-PoR scheme is secure if it is $\epsilon$-extractable, and satisfies verification correctness, $\Upsilon$-real-time detection, and fair payment properties.

\end{definition}




\begin{remark}
The folklore assumption is that (in a secure blockchain) a smart contract function \emph{always outputs a correct result}. However, this is not the case and it may fail under certain circumstances.  For instance, as shown in \cite{LuuTKS15} all rational  miners may not verify a certain transaction. As another example,  an adversary (although with a negligibly small probability)  discards a  certain honestly generated blocks,  reverses the state of blockchain and contract, or breaks a client's signature scheme.  Accordingly, in our definitions above, we take such cases  into consideration and allow the possibility that a function outputs an incorrect result even though with a negligibly small probability. 
\end{remark}




\begin{remark}
SO-PoR model differs from traditional (e.g. \cite{DBLP:conf/ccs/JuelsK07,DBLP:conf/asiacrypt/ShachamW08}) and outsourced PoR  (e.g. \cite{armknecht2014outsourced,xu2016lightweight}) models in several aspects. Only  the SO-PoR model offers all the properties. In particular,  traditional PoRs only offer extractability while outsourced ones  offer liability as well, that allows a client (by re-running all verifications function) to detect a verifier if it provides an incorrect verification output, so the client   cannot rely on the verification result provided.  As another difference, the SO-PoR model  takes into account the case where an adversary can corrupt both the server and some miners at the same time.
\end{remark}

\begin{remark}
SO-PoR should also support  the traditional PoR where only client and server interact with each other  (e.g. client generates challenges, and verifies proof) when the client is available. To let SO-PoR definition support that too, we can simply define a flag: $\xi$, in each function, such that  when $\xi=1$, it acts as the traditional PoR; otherwise (when $\xi=0$), it performs as a delegated one. For the sake of simplicity, we let the flag  be implicit in the definitions above, where the default  is  $\xi=1$ 
\end{remark}


\subsection{SO-PoR Protocol}\label{SO-PoR-Protocol}
This section presents SO-PoR protocol in detail, followed by the rationale behind it.      





\begin{enumerate}[leftmargin=.46cm]

\item\textit{\textbf{Client-side Setup}}. 



\begin{enumerate}
%\item Signs and deploys a smart contract: $\mathcal{SC}$ to a blockchain.

% where the contract contains a set of public parameters: e.g. $z$: total number of verifications,  $|F|$: file bit size, $\Delta_{\scriptscriptstyle 1}$:  maximum time period  taken by the server to generate a proof, $\Delta_{\scriptscriptstyle 2}$: time window in which a message is (sent by the server and) received by the contract. The client deposits $e\cdot z$  coins for $z$ successful  verifications. 

\item  \textbf{\textit{\small {Gen. Public and Private Keys}}}:   Picks a fresh key: $\hat{k}$ and two vectors of keys: $\vv{\bm{v}}$ and $\vv{\bm{l}}$, where each vector contains $z$ fresh keys. It picks a large prime number:  $p$ whose size is determined by a security parameter, i.e. $|p|=\iota$.  Moreover, it runs $\mathtt{Setup}(.)$ in  C-TLP scheme to generate a key pair: $(pk, sk)$

\item \textbf{\textit{\small {Gen. Other Public Parameters}}}:  Sets $c$ to the total number of blocks challenged in each verification. It defines  parameters: $w$ and $g$, where  $w$ is an index  of a future block: $\mathcal {B}_{\scriptscriptstyle w}$ in a blockchain that will be added to the blockchain (permanent state) at about the time  first delegated verification will  be done, and $g$ is  a security parameter referring to the number of blocks (in a row) starting from  $w$.  It  sets $z$: the total number of verifications,  $||{\bm{F}}||$: file bit size, $\Delta_{\scriptscriptstyle 1}$:  the maximum time  is taken by the server to generate a proof, $\Delta_{\scriptscriptstyle 2}$: time window in which a message is (sent by the server and) received by the contract, and $e$ amount of coins paid to the server for each successful  verification. Sets $\hat{pk}: (pk,e,g,w,p,c,z,\Delta_{\scriptscriptstyle 1},\Delta_{\scriptscriptstyle 2})$. 


\item\textbf{\textit{\small {Sign and Deploy   Smart Contract}}}: Signs and deploys a smart contract: $\mathcal{SC}$ to a blockchain.  It stores  public parameters: $(z,||{\bm{F}}||, \Delta_{\scriptscriptstyle 1},\Delta_{\scriptscriptstyle 2},c, g,p,w)$, on the contract. It deposits $e z$ coins to the contract. Then, it asks the server to sign the contract. The server signs if it agrees on all parameters.

\end{enumerate}
\item\textit{\textbf{Client-side Store}}.


\begin{enumerate}

\item \textbf{\textit{\small {Encode File}}}: Splits an error-corrected file, e.g. under Reed-Solomon codes, into $n$ blocks; ${\bm{F}}: [F_{\scriptscriptstyle 1},...,F_{\scriptscriptstyle n}]$,  where $ F_{\scriptscriptstyle i}\in \mathbb{F}_p$
\item\label{gen-client-server-tags}\textbf{\textit{\small {Gen. Permanent Tags}}}:  Using the key: $\hat{k}$, it computes $n$ pseudorandom values:  $r_{\scriptscriptstyle i}$ and single value: $\alpha$, as follows.  

 $$\alpha=\mathtt{PRF}(\hat{k},n+1)\bmod p$$
 $$ \forall i, 1\leq i\leq n: r_{\scriptscriptstyle i}=\mathtt{PRF}(\hat{k},i)\bmod p$$

 
 It uses the pseudorandom values to compute tags for the file blocks. 
 
 $$\forall i, 1\leq i\leq n: \sigma_{\scriptscriptstyle i}= r_{\scriptscriptstyle i}+ \alpha\cdot F_{\scriptscriptstyle  i}\bmod p$$
 
  So, at the end of this step,  a set of  tags are generated, $\sigma:\{\sigma_{\scriptscriptstyle 1},..., \sigma_{\scriptscriptstyle n}\}$
\item\label{Gen-Disposable-Tags}\textbf{\textit{\small {Gen. Disposable Tags}}}: For   $j\text{\small{-th}}$ verification  ($1\leq j\leq z$):
\begin{enumerate}
\item chooses the related key: $v_{\scriptscriptstyle j}\in\vv{\bm{v}}$ and computes $c$ pseudorandom indices. 

$$\forall b, 1\leq b\leq c: x_{\scriptscriptstyle b,j}=\mathtt{PRF}(v_{\scriptscriptstyle j}, b)\bmod n$$

\item picks the corresponding  key: $l_{\scriptscriptstyle j}\in \vv{\bm{l}}$ and computes $c$ pseudorandom values:  $r_{\scriptscriptstyle b,j}$ and single value: $\alpha_{\scriptscriptstyle j}$

  $$\alpha_{\scriptscriptstyle j}=\mathtt{PRF}(l_{\scriptscriptstyle j},c+1)\bmod p$$
  $$ \forall b, 1\leq b\leq c: r_{\scriptscriptstyle b,j}=\mathtt{PRF}(l_{\scriptscriptstyle j},b)\bmod p$$


\item generates $c$ disposable tags.  

$$\forall b, 1\leq b\leq c: \sigma_{\scriptscriptstyle b,j}=r_{\scriptscriptstyle b,j}+\alpha_{\scriptscriptstyle j}\cdot F_{\scriptscriptstyle y}\bmod p$$

  where $y= x_{\scriptscriptstyle b,j}$. At the end of this step, a set $\sigma_{\scriptscriptstyle j}$ of $c$ tags are computed,  $\sigma_{\scriptscriptstyle j}:\{\sigma_{\scriptscriptstyle 1,j},..., \sigma_{\scriptscriptstyle c,j}\}$

%\item encrypts $k_{\scriptscriptstyle 1}^{\scriptscriptstyle j}$ using the time-lock encryption, such that it can be decrypted at time $t^{\scriptscriptstyle j}_{\scriptscriptstyle 1}$, i.e., $C^{\scriptscriptstyle j}_{\scriptscriptstyle 1}=\mathcal{ENC}^{\scriptscriptstyle pk}_{\scriptscriptstyle sk,T}(k_{\scriptscriptstyle 1}^{\scriptscriptstyle j})$, where $T=t^{\scriptscriptstyle j}_{\scriptscriptstyle 1}\cdot S$, and $S$ is a parameter of the encryption: the number of squaring modulo $N$ per second that can be performed by a solver. Also, it uses the time-lock encryption to encrypt $k^{\scriptscriptstyle j}_{\scriptscriptstyle 2}$, such that it can be decrypted at time $ t^{\scriptscriptstyle j}_{\scriptscriptstyle 2}$,  i.e., $C^{\scriptscriptstyle j}_{\scriptscriptstyle 2}=\mathcal{ENC}^{\scriptscriptstyle pk}_{\scriptscriptstyle sk,T'}(k_{\scriptscriptstyle 2}^{\scriptscriptstyle j})$, where $T'= t^{\scriptscriptstyle j}_{\scriptscriptstyle 2}\cdot S$ and $t^{\scriptscriptstyle j}_{\scriptscriptstyle 2}> t^{\scriptscriptstyle j}_{\scriptscriptstyle 1}+\Delta_{\scriptscriptstyle 1}+\Delta_{\scriptscriptstyle 2}$,  $\Delta_{\scriptscriptstyle 1}$  is the time period within which some local computation (by the server) is performed on $k^{\scriptscriptstyle j}_{\scriptscriptstyle 1}$ and $\Delta_{\scriptscriptstyle 2}$ is the time window in which a message, e.g. $k^{\scriptscriptstyle j}_{\scriptscriptstyle 1}$, is (sent by the server and) received by the contract. Note that, the size of time windows are sufficiently large.

%\item\label{gen-hashes} computes $h_{\scriptscriptstyle j}= H(l_{\scriptscriptstyle j})$.


%$\resizeT{\textit{w}}_{\resizeS {\textit  j}}$

%\resizeS {\textit  w}_{\resizeS {\textit  j}}}

\end{enumerate} 

\item\label{Gen-Puzzles-}\textbf{\textit{\small {Gen. Puzzles}}}: Sets $\vv{\bm{m}}=[v_{\scriptscriptstyle 1},l_{\scriptscriptstyle 1},...,v_{\scriptscriptstyle z},l_{\scriptscriptstyle z}]$  and then encrypts the vector's elements, by running: $\mathtt{GenPuz}(\vv{\bm{m}},pk,sk)$ in   C-TLP scheme. This yields a  puzzle vector: $[(V_{\scriptscriptstyle 1},L_{\scriptscriptstyle 1}),...,(V_{\scriptscriptstyle z},L_{\scriptscriptstyle z})]$ and a commitment vector $\vv{\bm{h}}$. The encryption is done in  such a way that in each $j\text{\small{-th}}$ pair, $V_{\scriptscriptstyle j}$ will be fully decrypted at times $t_{\scriptscriptstyle j}$ and $L_{\scriptscriptstyle j}$ will be decrypted at time $t'_{\scriptscriptstyle j}$, where  $ t_{\scriptscriptstyle j}+\Delta_{\scriptscriptstyle 1}+\Delta_{\scriptscriptstyle 2}\leq t'_{\scriptscriptstyle j} < t_{\scriptscriptstyle j+1}$  %$\Delta_{\scriptscriptstyle 1}$  is the maximum  period  the server needs to generate a proof and $\Delta_{\scriptscriptstyle 2}$ is the time window in which a message is (sent by the server and) received by the contract.




%\item Using  C-TLP scheme, generates a key pair: ($sk_{\scriptscriptstyle 1}, pk_{\scriptscriptstyle 1}$) and  encrypts    $[v_{\scriptscriptstyle 1},...,v_{\scriptscriptstyle z}]$ such that they will be fully decrypted at times $[t_{\scriptscriptstyle 1},...,t_{\scriptscriptstyle z}]$ respectively. This yields a  ciphertext vector: $[V_{\scriptscriptstyle 1},...,V_{\scriptscriptstyle z}]$. Invoking  C-TLP scheme again, it generates key pairs: ($sk_{\scriptscriptstyle 2}, pk_{\scriptscriptstyle 2}$) and  encrypts    $[l_{\scriptscriptstyle 1},...,l_{\scriptscriptstyle z}]$ that will be fully decrypted at times $[t'_{\scriptscriptstyle 1},...t'_{\scriptscriptstyle z}]$ respectively.  This yields a  ciphertext vector: $[L_{\scriptscriptstyle 1},...,L_{\scriptscriptstyle z}]$. Note,   $t'_{\scriptscriptstyle j}\geq t_{\scriptscriptstyle j}+\Delta_{\scriptscriptstyle 1}+\Delta_{\scriptscriptstyle 2}$, where $\Delta_{\scriptscriptstyle 1}$  is the maximum time period  the server needs to generate a proof and $\Delta_{\scriptscriptstyle 2}$ is the time window in which a message is (sent by the server and) received by the contract.

%\item And another one  with input messages  $l_{\scriptscriptstyle 1},...,l_{\scriptscriptstyle z}$ that will be decrypted at times $t'_{\scriptscriptstyle 1},...t'_{\scriptscriptstyle z}$ respectively, where  $t'_{\scriptscriptstyle j}=t_{\scriptscriptstyle j}+\Delta_{\scriptscriptstyle 1}+\Delta_{\scriptscriptstyle 2}$,  $\Delta_{\scriptscriptstyle 1}$  is the time period within which some local computation (by the server) is performed on $v_{\scriptscriptstyle j}$ and $\Delta_{\scriptscriptstyle 2}$ is the time window in which a message, e.g. $v_{\scriptscriptstyle j}$, is (sent by the server and) received by the contract. %Note that, the size of time windows are sufficiently large.



\item\label{Outsource-File}\textbf{\textit{\small {Outsource File}}}: Stores ${\bm{F}},n,\hat{pk}, \{\sigma,\sigma_{\scriptscriptstyle 1},..., \sigma_{\scriptscriptstyle z}, (V_{\scriptscriptstyle 1},L_{\scriptscriptstyle 1}),...,(V_{\scriptscriptstyle z},L_{\scriptscriptstyle z})\}$   on the server. Also, it stores $\vv{\bm{h}}$ on the smart contract. 
\end{enumerate}


\item\textit{\textbf{Cloud-Side Proof Generation}}. For   $j\text{\small{-th}}$ verification  ($1\leq j\leq z$), the cloud:


\begin{enumerate} 
\item\label{Solve-Puzzle-Regen-Indices}\textbf{\textit{\small {Solve Puzzle and Regen.  Indices}}}.   Receives and parses the output of $\mathtt{SolvPuz}(.)$ in C-TLP, to extract $v_{\scriptscriptstyle j}$, at time $t_{\scriptscriptstyle j}$. Next, using $v_{\scriptscriptstyle j}$, it regenerates $c$ pseudorandom indices. 

$$\forall b, 1\leq b\leq c: x_{\scriptscriptstyle b,j}=\mathtt{PRF}(v_{\scriptscriptstyle j}, b)\bmod n$$ %where $v_{\scriptscriptstyle j}$ is the key fully decrypted by the cloud at time $t_{\scriptscriptstyle j}$ for this verification.


\item \textbf{\textit{\small {Extract Key}}}. Extracts a seed: $u_{\scriptscriptstyle j}$, from the blockchain as follows: $u_{\scriptscriptstyle j}=\mathtt{H}( \mathcal {B}_{\scriptscriptstyle \gamma}||...||  \mathcal {B}_{\scriptscriptstyle \zeta})$, where $\gamma=w+(j-1)\cdot g$ and $\zeta=w+j\cdot g$

\item\label{Gen-PoR}\textbf{\textit{\small {Gen. PoR}}}. Generates a PoR proof. 

$$\mu_{\scriptscriptstyle j}=\sum\limits^{\scriptscriptstyle c}_{\scriptscriptstyle b=1}  \mathtt{PRF}(u_{\scriptscriptstyle j},b)\cdot F_{\scriptscriptstyle y}\bmod p, \  \  \ \xi_{\scriptscriptstyle j}= \sum\limits^{\scriptscriptstyle c}_{\scriptscriptstyle b=1}  \mathtt{PRF}(u_{\scriptscriptstyle j},b)\cdot \sigma_{\scriptscriptstyle b,j}\bmod p$$
 
 where  $y$ is a pseudorandom index: $y= x_{\scriptscriptstyle b,j}$ %Also, it runs $\mathtt{Prove}(.)$ in C-TLP, to generate a proof: $\ddot{p}_{\scriptscriptstyle j}$, of $v_{\scriptscriptstyle j}$'s correctness
 
 \item\label{Register-Proofs}\textbf{\textit{\small {Register Proofs}}}.  Sends the PoR proof: $(\mu_{\scriptscriptstyle j},\xi_{\scriptscriptstyle j})$   to the smart contract within  $\Delta_{\scriptscriptstyle1}$
 \item\label{fully-recover-l}\textbf{\textit{\small {Solve Puzzle and Regen.  Verification Key}}}: Receives and parses the output of $\mathtt{SolvPuz}(.)$ in C-TLP to extract $l_{\scriptscriptstyle j}$, at time $t'_{\scriptscriptstyle j}$. Also, it runs $\mathtt{Prove}(.)$ in C-TLP, to generate a proof: $\ddot{p}_{\scriptscriptstyle j}$, of $l_{\scriptscriptstyle j}$'s correctness. It sends $\ddot{p}_{\scriptscriptstyle j}$ (containing $l_{\scriptscriptstyle j}$)  to the contract, so it can be received by the contract within $\Delta_{\scriptscriptstyle 2}$ 
\end{enumerate}


\item \textit{\textbf{Smart Contract-Side Verification}}. For   $j\text{\small{-th}}$ verification  ($1\leq j\leq z$), the contract:



\begin{enumerate} 
\item\textbf{\textit{\small {Check Arrival Time}}}: checks the arrival time of the decrypted values sent by the server. In particular, it checks, if $(\mu_{\scriptscriptstyle j},\xi_{\scriptscriptstyle j})$ was received in the time window: $(t_{\scriptscriptstyle j}, t_{\scriptscriptstyle j}+\Delta_{\scriptscriptstyle 1}+\Delta_{\scriptscriptstyle 2}]$ and whether $l_{\scriptscriptstyle j}$ was received in the time window: $(t'_{\scriptscriptstyle j}, t'_{\scriptscriptstyle j}+\Delta_{\scriptscriptstyle 2}]$

\item\label{check-hash}\textbf{\textit{\small {Verify Puzzle Solution}}}: runs $\mathtt{Verify}(.)$ in C-TLP to verify $\ddot{p}_{\scriptscriptstyle j}$  (i.e. check the correctness of $l_{\scriptscriptstyle j}\in \ddot{p}_{\scriptscriptstyle j}$). If approved, then regenerates the seed:  $u_{\scriptscriptstyle j}=\mathtt{H}( \mathcal {B}_{\scriptscriptstyle \gamma}||...||  \mathcal {B}_{\scriptscriptstyle \zeta})$, where $\gamma=w+(j-1)\cdot g$ and $\zeta=w+j\cdot g$


%$s_{\scriptscriptstyle j}=H( \mathcal {B}_{\resizeS {\textit  w}_{\resizeSS {\textit  j}}}||,...,||  \mathcal {B}_{\resizeS {\textit  w}_{\resizeSS {\textit  j}}+g})$, given parameters $g$ and $\resizeT{\textit{w}}_{\resizeS {\textit  j}}$. 


\item\label{verify-PoR}\textbf{\textit{\small {Verify PoR}}}: regenerates the pseudorandom values and verifies the PoR proof.  
\begin{equation}\label{POR-ver}\xi_{\scriptscriptstyle j}\stackrel{\scriptscriptstyle ?}=\mu_{\scriptscriptstyle j}  \cdot\mathtt{PRF}(l_{\scriptscriptstyle j},c+1)+\sum\limits^{\scriptscriptstyle c}_{\scriptscriptstyle b=1} ( \mathtt{PRF}(u_{\scriptscriptstyle j},b)\cdot \mathtt{PRF}(l_{\scriptscriptstyle j},b))\bmod p
\end{equation}
\item\textbf{\textit{\small {Pay}}}: if  Equation \ref{POR-ver} holds, pays and asks the server to delete all disposable tags for this verification, i.e. $\sigma_{\scriptscriptstyle j}$
\end{enumerate}
If either check fails, it aborts and notifies the client. 


\item \textit{\textbf{Client-server PoR}}: When the client is online, it can   interact  with the server  to check its data availability. In particular, it sends $c$ random challenges and random indices to the server who computes POR using only: (a) the  messages sent by the client in this step, (b) the file: $\bm{F}$, and (c) the tags:  $\sigma_{\scriptscriptstyle i}\in\sigma$, generated in step \ref{gen-client-server-tags}.  The proof generation and verification are similar to the MAC-based schemes, e.g.  \cite{DBLP:conf/asiacrypt/ShachamW08}. 
\end{enumerate}


% !TEX root =main.tex

\section{Further Remarks on SO-PoR Protocol}\label{SO-PoR-discussion}





\begin{remark}
 In each verification, e.g. $j\text{\small{-th}}$ one, it is required that the server can: (a) learn the random challenges, (b) compute a proof, and (c) record it in the smart contract, before it is able to learn key $l_{\scriptscriptstyle j}$; otherwise, (i.e. if it learns $l_{\scriptscriptstyle j}$ before sending and registering the proof), it can tamper with the data and pass the verification. Because by knowing $l_{\scriptscriptstyle j}$ it can construct valid tags for the data that has been tampered with. That is why, in the protocol, it is required: $t'_{\scriptscriptstyle j}\geq t_{\scriptscriptstyle j}+\Delta_{\scriptscriptstyle 1}+\Delta_{\scriptscriptstyle 2}$
\end{remark}


\begin{remark}
The way disposable tags are generated in SO-PoR  differs from those computed  in traditional/outsourced PoR schemes, in spite of having similarities structure-wise. Specifically, (unlike existing protocols, e.g. \cite{DBLP:conf/asiacrypt/ShachamW08,armknecht2014outsourced}) in SO-PoR, each random value, $r_{\scriptscriptstyle b,j}$, utilised to generate  a disposable tag of a  block (for $j\text{\small{-th}}$ verification), is not derived from the block index. Instead, it depends on (a  fresh secret key for $j\text{\small{-th}}$ verification and) the total number of blocks challenged in each verification that is a public value. This means, the verifier does not need to know and verify each challenged block's index in the verification phase, which leads to a lower cost.  
\end{remark}

\begin{remark}
With minor adjustments, we can reduce the smart contract storage cost from $O(z)$ to constant, $O(1)$ and offload the cost to the server. The idea is that the client after computing the commitmnet vector: $\vv{\bm{h}}=[h_{\scriptscriptstyle 1},...h_{\scriptscriptstyle z}]$,  in step \ref{Gen-Puzzles-}, it preserves the ordering of the elements (i.e. $h_{\scriptscriptstyle j}$ is associated with $j\text{\small{-th}}$ verification) and constructs a  Merkle tree  on top of them. It stores the tree and the vector on the server, and stores only the tree's  root: $R$, on the contract. In this case,  the server in step \ref{fully-recover-l} after recovering $\ddot{p}_{\scriptscriptstyle j}= (l_{\scriptscriptstyle j}, d_{\scriptscriptstyle j})$,  computes: $h_{\scriptscriptstyle j}=\mathtt{H}(l_{\scriptscriptstyle j}||d_{\scriptscriptstyle j})$, and sends a Merkle tree proof (that $h_{\scriptscriptstyle j}$ corresponds to  $R$) along with $\ddot{p}_{\scriptscriptstyle j}$ to the contract. In step \ref{check-hash}, the contract: (a) checks if $h_{\scriptscriptstyle j}=\mathtt{H}(l_{\scriptscriptstyle j}||d_{\scriptscriptstyle j})$, and  (b) verifies the Merkle tree proof.  The rest  remains unchanged.  As a result, the number of values stored in the contract is now $O(1)$. This adjustment comes with an added communication cost: $O(|h_{\scriptscriptstyle j}|\log z)$ for each verification. Nevertheless, the added cost is small and independent of the file size.   For instance, when  $z=10^{\scriptscriptstyle 6}$ and $|h_{\scriptscriptstyle j}|=256$, the  added communication cost is only about $5.1$ kilobit.

\end{remark}

\begin{remark}
One might be willing to use a combination of existing publicly verifiable PoR and smart contract, such that the contract performs the verification on the client's behalf. However, this approach would have a higher computation or communication cost (especially in the verification phase) than our protocol. Specifically,  there exist two publicly verifiable PoR schemes, based on either (a) BLS signatures, e.g. \cite{DBLP:conf/asiacrypt/ShachamW08}, or (b) Merkle tree, e.g. \cite{MillerPermacoin}. The former approach, in total, requires  $zc$ exponentiations in the verification phase, whereas SO-PoR requires no exponentiations in this phase. Also, the BLS signature-based scheme takes a very long time to encode  a file, as the required number of exponentiations is  $O(|{\bm{F}}|)$. For instance, as measured in \cite{armknecht2014outsourced}, it takes about $55$ minutes to encode a  $64$-MB file,  meaning  it would take about $14$ hours to encode a $1$-GB file. But, in SO-PoR the number of exponentiations, in the store phase, is independent of and much fewer than the file size. On the other hand, the proof size in the Merkle tree-based approach is logarithmic with the file size, i.e. $256zc\log |{\bm{F}}|$ bits, that leads to a high server-side's communication cost. By contrast,  the proof size in SO-PoR is  independent of the file size and is much shorter, i.e. $884z$ bits. 


\end{remark}


% !TEX root =main.tex


\subsection{SO-PoR Security Proof}\label{SO-PoR-Security-Proof}
In this section, we first present the main  theorem of SO-PoR protocol and then prove it.  





\begin{theorem} SO-PoR protocol is secure w.r.t. Definition \ref{SO-PoR-Security} if the tags/MAC's are unforgeable, $\mathtt{PRF}(.)$ is a secure pseudorandom function, the blockchain is secure, C-TLP protocol is secure, and $\mathtt{H}( \mathcal {B}_{\scriptscriptstyle \gamma}||...||  \mathcal {B}_{\scriptscriptstyle \zeta})$ outputs an unpredictable random value (where $\zeta-\gamma$ is a security parameter).
\end{theorem}

\begin{proof}[sketch]  In the following, we prove that SO-PoR protocol satisfies every  property defined in Section \ref{SO-PoR-Model}.   

\

\noindent\textbf{\textit{Verification correctness}} (according to Definition \ref{Verification-Correctness}). We first argue that the adversary who corrupts either $C\subseteq\{\mathcal{M}_{\scriptscriptstyle 1},...,\mathcal{M}_{\scriptscriptstyle\beta}\}$ or $C'\subseteq\{\mathcal{S},\mathcal{M}_{\scriptscriptstyle 1},...,\mathcal{M}_{\scriptscriptstyle\beta-1}\}$ with a high probability, cannot influence the output of $\mathtt{Verify}(.)$ performed by a smart contract in a blockchain; in other words, the verification correctness holds. In short, the verification correctness boils down to the security of the underlying blockchain. In the case where the adversary corrupts $C$ (when the server provides an accepting proof),  for the adversary to make the verification function output $0$, it has to: (a) either forge the server's signature, or (b)  fork the blockchain so the chain  comprising the accepting proof is discarded. In case (a),  if it manages to forge the signature, it can generate a transaction that includes   a rejecting proof where the transaction is signed on the server's behalf. In this case, it  can broadcast the transaction as soon as the transaction containing an accepting proof is broadcast,  to make the latter transaction stale. Nevertheless, the probability of such  an event is negligible, $\epsilon(\lambda')$, as long as the signature is secure. Moreover, due to the liveness property of blockchain, an honestly generated transaction will eventually appear on an honest miner's chain \cite{DBLP:conf/crypto/GarayKL17}. In case (b),  the adversary has to generate long enough (valid) chain that excludes the accepting proof, but this also has a negligible success probability, $\epsilon(\lambda')$, under the assumption that the hash power of the adversary is lower than those of  honest miners (i.e. under the honest majority assumption) and due to the liveness property. Now we turn our attention to the case where the adversary corrupts $C'$ (when the server provides a rejecting proof). In this case, for the adversary to make the verification function to output $1$, the honest miners must not validate the transaction that contains the proof. Nevertheless, as long as the blockchain is secure and  the computational advantage of skipping transaction validation is low, i.e. the validation imposes a low computation cost, the miners check   the transaction's validation \cite{LuuTKS15}. Also, as shown in \cite{LuuTKS15} when a transaction validation imposes a high computation cost, two generic techniques can be used to support exact or probabilistic correctness (of a smart contract function output). We conclude the correctness of $\mathtt{Verify}(.)$ output  is guaranteed with a high probability.

\

\noindent\textbf{\textit{$\epsilon$-extractable}} (according to Definition \ref{extractable}). In the following, we show that if   a proof produced by an adversary: $\mathcal{A}$ who corrupts $C'\subseteq\{\mathcal{S},\mathcal{M}_{\scriptscriptstyle 1},...,\mathcal{M}_{\scriptscriptstyle\beta-1}\}$ is accepted by $\mathtt{Verify(.)}$ with probability at least $\epsilon$, then the file can be extracted by a means of an extraction algorithm. As mentioned  before, $\mathtt{Verify(.)}$ can use both disposable and permanent tags, in the latter case $\mathtt{Verify(.)}$ is run by a smart contract, while in the former one the client runs it. For the sake of simplicity, we first consider the case where $C'=\mathcal{S}$. In this case, the extractability proof is similar to the one in \cite{DBLP:conf/asiacrypt/ShachamW08},  with a few differences, in SO-PoR: (a) the extractor can use both disposable tags and  permanent tags (when the former run out), (b) assumes  C-TLP protocol is secure, (c) assumes $\mathtt{H}( \mathcal {B}_{\scriptscriptstyle \gamma}||...||  \mathcal {B}_{\scriptscriptstyle \zeta})$ outputs a random value  even if $\beta$ miners try to influence its output. Note that in \cite{DBLP:conf/asiacrypt/ShachamW08} only the permanent tags are used, and since  the client generates the challenges and performs the verification, it does not use other primitives; hence, it does not require other security assumptions. As proven in  Theorems \ref{Solution-Privacy} and \ref{Solution-Validity}, C-TLP protocol is secure. Moreover, as  analysed and proven in \cite{DBLP:journals/iacr/AbadiCKZ19,armknecht2014outsourced}, an output of $\mathtt{H}( \mathcal {B}_{\scriptscriptstyle \gamma}||...||  \mathcal {B}_{\scriptscriptstyle \zeta})$ is random value even if some blocks are generated or selectively disseminated by malicious miners. We conclude that the extractor can extract the file when $C'=\mathcal{S}$, and  the extractor is interacting a  $\epsilon$-admissible prover. Now we move on to the case where $C'\subseteq\{\mathcal{S},\mathcal{M}_{\scriptscriptstyle 1},...,\mathcal{M}_{\scriptscriptstyle\beta-1}\}$. In this case, the proof (provided by $\mathcal{S}$) is rejecting but the corrupt miners may try to make $\mathtt{Verify}(.)$ output $1$. Note,  if they succeed to do so, then the file can be  extracted only by using the permanent tags but not the disposable ones. However, as shown above (i.e. due to the verification correctness), they have a negligible probability of success, when the blockchain is secure. 

\

\noindent\textbf{\textit{$\Upsilon$-real-time Detection}} (according to Definition \ref{real-time Detection}). In the following, we argue that after the server broadcasts a proof at a certain time, say $t$, to the network, the client can get a correct output of $\mathtt{Verify}(.)$ at most after time period $\Upsilon$, by a means of  reading the blockchain. The proof  is split into two parts: (a) correctness of $\mathtt{Verify}(.)$ output, and (b) the maximum delay on the client's view of the output. Since we have already shown above that (when $C$ or $C'$ is corrupt) the correctness of $\mathtt{Verify}(.)$  output is guaranteed,  we focus on the latter property, i.e. the delay. To describe the delay, we need to recall two blockchain notions: \emph{liveness} and \emph{slackness} \cite{DBLP:conf/crypto/GarayKL17,DBLP:conf/crypto/BadertscherMTZ17}. Informally, liveness states that an honestly generated transaction  will eventually be included more than $\mathtt{k}$ blocks deep in an honest party's  blockchain \cite{DBLP:conf/crypto/GarayKL17}. It is parameterised by wait time: $\mathtt{u}$ and depth: $\mathtt{k}$. We can fix the parameters as follows. We set $\mathtt{k}$ as the minimum depth of a block  considered as the blockchain's \emph{state}  (i.e. a part of the blockchain that remains unchanged with a high probability, e.g. $\mathtt{k}\geq 6$) and $\mathtt{u}$ the waiting time  that the transaction gets $\mathtt{k}$ blocks deep. As shown in \cite{DBLP:conf/crypto/BadertscherMTZ17}, there is a slackness on  honest parties' view of the blockchain.  In particular, there is no guarantee that at any given time, all honest miners have the same view of the blockchain, or even the  state. But, there is an upper-bound on the slackness,  denoted by $\mathtt{WindowSize}$, after which all honest parties would have the same view on a certain part of the blockchain state. This means when an honest party (e.g. the server) propagates its transaction (containing the proof) all honest parties will see it on their chain after at most: $\Upsilon=\mathtt{WindowSize}+\mathtt{u}$ time period. So when  the adversary corrupts $C$ or $C'$, but in the latter case the server constructs a valid transaction (regardless of the proof status) the client by reading the blockchain (i.e. probing the miners) can get a correct result after at most time period $\Upsilon$ when the server sends the proof. Also, when parties in $C'$ are corrupt and the transaction (containing the proof) is not valid,  as discussed above  (for verification correctness) the honest miners would detect the invalid transaction and do not include in the chain, therefore the output of $\mathtt{Verify(.)}$ would be the same as its default value: $0$; the same holds  when the server sends nothing to the network. This concludes the proof related to $\Upsilon$-real-time detection in SO-PoR protocol.


\

\noindent\textbf{\textit{Fair Payment}} (according to Definition \ref{Fair-Payment}).  The proof  takes into consideration that the correctness of $\mathtt{Verify}(.)$ output is guaranteed (as shown above). It boils down to the correctness of $\mathtt{Pay}(.)$ as it is interrelated to $\mathtt{Verify}(.)$ output. In particular, for the adversary to make  inequality \ref{inequ::fair-payment} not  hold, it has to break $\mathtt{Pay}(.)$ correctness, e.g. pays the server despite the verification function outputs $0$. Therefore, it would suffice to show the adversary who corrupts $C$ or $C'$ cannot affect $\mathtt{Pay}(.)$ output's correctness. To do so, we can apply the same argument used to prove  the correctness of $\mathtt{Verify}(.)$ above, as the correctness of  $\mathtt{Pay}(.)$ relies on the security of the blockchain as well.  \hfill\(\Box\)
\end{proof}


%
%\begin{theorem}\label{PoR-main-theorem} SO-PoR protocol is secure  if the MAC's are unforgeable, $\mathtt{PRF}(.)$ is a secure pseudorandom function, the blockchain and C-TLP protocol are secure, and $\mathtt{H}( \mathcal {B}_{\scriptscriptstyle \gamma}||...||  \mathcal {B}_{\scriptscriptstyle \zeta})$ outputs an unpredictable random value (where $\zeta-\gamma$ is a security parameter).
%\end{theorem}
%
%\begin{remark}
%We refer readers to Appendix \ref{SO-PoR-Security-Proof} for the above theorem's proof.
%\end{remark}
%
%
%
%\begin{remark}
%In SO-PoR, for a security reason the server must record $j$-th PoR proof in the contract before $l_{j}$ is recovered. Also,  the way disposable tags are generated in SO-PoR  differs from those computed  in previous PoR schemes, in spite of having similarities structure-wise. Moreover, with slight adjustments, we can reduce the contract-side storage cost to constant.  For more details, we refer readers to Appendix \ref{SO-PoR-discussion} that also explains why strawman approaches are not suitable substitutes for SO-PoR. 
%\end{remark}
%\vspace{-3mm}
%\subsection{Extension: Reducing Smart Contract Storage Cost to Constant}\label{storage-cost-reduction}
%
%
%With minor adjustments, we can reduce the smart contract storage cost from $O(z)$ to constant, $O(1)$ and offload the cost to the server. The idea is that the client after computing the commitmnet vector: $\vv{\bm{h}}=[h_{\scriptscriptstyle 1},...h_{\scriptscriptstyle z}]$,  in step \ref{Gen-Puzzles-}, it preserves the ordering of the elements (i.e. $h_{\scriptscriptstyle j}$ is associated with $j^{\scriptscriptstyle th} $ verification) and constructs a  Merkle tree  on top of them. It stores the tree and the vector on the server, and stores only the tree's  root: $R$, on the contract. In this case,  the server in step \ref{fully-recover-l} after recovering $\ddot{p}_{\scriptscriptstyle j}= (l_{\scriptscriptstyle j}, d_{\scriptscriptstyle j})$,  computes: $h_{\scriptscriptstyle j}=\mathtt{H}(l_{\scriptscriptstyle j}||d_{\scriptscriptstyle j})$, and sends a Merkle tree proof (that $h_{\scriptscriptstyle j}$ corresponds to  $R$) along with $\ddot{p}_{\scriptscriptstyle j}$ to the contract. In step \ref{check-hash}, the contract: (a) checks if $h_{\scriptscriptstyle j}=\mathtt{H}(l_{\scriptscriptstyle j}||d_{\scriptscriptstyle j})$, and  (b) verifies the Merkle tree proof.  The rest  remains unchanged.  As a result, the number of values stored in the contract is now $O(1)$. This adjustment comes with an added communication cost: $O(|h_{\scriptscriptstyle j}|\log z)$ for each verification. Nevertheless, the added cost is small and independent of the file size.   For instance, when  $z=10^{\scriptscriptstyle 6}$ and $|h_{\scriptscriptstyle j}|=256$, the  added communication cost is only about $5.1$ kilobit.



%\subsection{Evaluation}
%
%In this section, we provide a summary of comparisons between   SO-PoR and outsourced PoRs  \cite{armknecht2014outsourced,xu2016lightweight,Storage-Time}. Among the two protocol in \cite{Storage-Time} we only consider ``basic PoSt'' as it supports public verifiability. Briefly, in terms of property, only SO-PoR offers an explicit solution for real-time detection and fair payment. In terms of computation cost, the verification algorithm in SO-PoR is much faster than other three protocols; Specifically, when $c=460$,  SO-PoR verification requires about $4.5$ times fewer computation than the verification required in the fastest outsourced PoR \cite{armknecht2014outsourced}.   Also, \cite{armknecht2014outsourced} has the worst store cost, which is  much higher than that of SO-PoR; e.g. for a $1$-GB file, SO-PoR requires over $46 \times 10^{\scriptscriptstyle 5}$ times fewer exponentiations than \cite{armknecht2014outsourced} needs in the same phase.  SO-PoR and \cite{Storage-Time} require a server to solve puzzles but the other two protocols do not need that. Also, the I/O cost and proof complexity of all protocols are $O(1)$ except \cite{Storage-Time} whose I/O cost and proof complexity are $O(\log n)$. The server-side bandwidth of SO-PoR is much lower than the rest;  for instance, for $1$-GB file and $z=100$ verifications, a server in SO-PoR requires $9\times 10^{\scriptscriptstyle4}$,  $7$ and $1729$ times fewer bits  than those required in \cite{armknecht2014outsourced}, \cite{xu2016lightweight} and \cite{Storage-Time} respectively.  A client in SO-PoR has a higher bandwidth than the rest of the protocols (but this cost is one-off). Thus, SO-PoR offers additional properties, it has lower verification cost and lower server-side bandwidth than the rest while its other costs remain reasonable. Tables \ref{table::O-PoR-Cost} outlines the cost comparison results. For a  full analysis, we refer readers to Appendix \ref{Full-Evaluation} where  we also compare SO-PoR costs with the costs of the most efficient  traditional PoR  \cite{DBLP:conf/asiacrypt/ShachamW08}.
%


% !TEX root =main.tex


\section{SO-PoR's Full Evaluation}\label{Full-Evaluation}

In the following, we  provide a full analysis of SO-PoR and  compare its properties and detailed costs to those protocols that support outsourced PoR (O-PoR), i.e. \cite{armknecht2014outsourced,xu2016lightweight,Storage-Time}. 
Note, there are two protocols  proposed in  \cite{Storage-Time}; in the following, we only consider the one that supports public verifiability, i.e. basic PoSt. Recall,  we consider a generic case where a client outsources $z$ verifications and  in our cost analysis, we also compare SO-PoR cost with the cost of the most efficient privately verifiable PoR \cite{DBLP:conf/asiacrypt/ShachamW08}, too. 

%The basic PoSt in \cite{Storage-Time} supports public verification; that means a smart cotnract can play the role of a validator. 


\noindent\textbf{\textit{Properties}}. We start with a crucial feature that any O-PoR must have: real-time detection. Recall,  real-time detection requires a client to receive a correct verification result in (almost)  real-time without the need for it to re-execute the verification itself. This is offered only by SO-PoR. By contrast, in \cite{armknecht2014outsourced}  the auditor may never notify the  client, even if it does, its notification would not be reliable, and the client has to redo the verification to verify the auditor's claim. Similarly, in \cite{xu2016lightweight} the client has to fully trust the auditor to get notified on-time. The basic PoSt in \cite{Storage-Time} requires the server to collect all PoR's and send them to a validator in one go. This means, the client and validator cannot detect data tampering in real-time if a subset of the PoR's are invalid; instead they need to wait until $z$ PoR's are collected by the server.  So, \cite{armknecht2014outsourced,xu2016lightweight,Storage-Time} are not suitable for the cases where a client must be notified by a potentially malicious auditor as soon as an unauthorised modification on the sensitive data is detected.  The  fair payment is another vital property in O-PoR, as the cloud server  and auditor must be paid fairly, in the \emph{real world} when they serve a client. This feature is explicitly captured by  only SO-PoR.  The protocols in \cite{armknecht2014outsourced,xu2016lightweight} do not have any mechanism in place. In \cite{armknecht2014outsourced}, one may allow  the auditor to pay the server on the client's behalf. But, this is problematic.   The server and auditor can collude to save costs, in a way that the server  generates accepting proofs for the client but generates no proof for the auditor, and still the auditor pays it. This violates the fair payment and cannot be detected by the client unless it performs all the verification itself. On the other hand, a client in \cite{xu2016lightweight}  has to fully trust the auditor (with the payment too), otherwise the auditor can collude with the server to violate the fair payment. The authors of \cite{Storage-Time}  briefly state that the basic PoSt's verification can be performed by a smart contract who, after ensuring the proofs are valid,  pays the server. Nevertheless, as stated previously, the verification cost of this protocol is too high for a smart contract, i.e. imposes at least $z$ modular exponentiations over RSA modulus and requires a high number of messages logarithmic with the file size to be sent to the contract. Thus, even though it can support fair payment \emph{in theory}, it is very costly in practice.  Another  property is the cost of onboarding a new verifier, as it determines how flexible the client can be, to pick a new auditor when its current one is misbehaving. This cost in \cite{armknecht2014outsourced} is significantly  high, as it requires the verifier to download the entire file, generate metadata, and prove in zero-knowledge the correctness of metadata to the client. But, that cost in SO-PoR and \cite{xu2016lightweight,Storage-Time} is very low as the client only sends them a small set of parameters without the need to access the outsourced data.  Furthermore, as stated above, a client in  \cite{xu2016lightweight} has to fully trust the auditor with the correctness of verification but this is not the case in   SO-PoR and \cite{armknecht2014outsourced,Storage-Time}, as they consider a potentially malicious auditor (under different assumptions).  Table \ref{table::O-PoR-Property} summaries the result of  the comparison between the four protocols' main properties. 

% !TEX root =main.tex


 \begin{table*}[htb]
%\begin{footnotesize}
\begin{center}
\caption{ \small{O-PoR's Properties Comparison. $\checkmark^{*}$ indicates the property is met only in theory.}} \label{table::O-PoR-Property} 
%\renewcommand{\arraystretch}{.9}
%\scalebox{0.98}{
\begin{tabular}{|c|c|c|c|c|c|c|c|c|c|c|c|c|c|} 
   \hline
\cellcolor[gray]{0.9}&
 \multicolumn{3}{c|}{\cellcolor[gray]{0.9}\scriptsize  \underline{\ \ \ \ \ \ \ \ \ \ \ \ \ \ \ \ \ \ \ \ \ \ \ \ \ \ \ \ \ \ \ \ \ \ \ \ \ \ \ \ \ \ \ \ \ \   \ \ \ \ \ Properties   \ \ \ \ \ \ \ \ \ \ \ \ \ \ \ \ \ \ \ \ \ \ \ \ \ \ \ \ \ \ \ \ \ \ \ \ \ \ \ \ \ \ \ \   \ \ \ }}\\
 %\cline{2-5}
 \cellcolor[gray]{0.9}\multirow{-2}{*} {\scriptsize Protocols}&\cellcolor[gray]{0.9}\scriptsize Real-time Detection&\cellcolor[gray]{0.9}\scriptsize Fair Payment&\cellcolor[gray]{0.9}\scriptsize Untrusted Auditor\\
\hline
 \cellcolor[gray]{0.9}{\scriptsize  SO-PoR }&\scriptsize$\checkmark$&\scriptsize$\checkmark$&\scriptsize$\checkmark$\\
    
     \cline{2-4}    
     \hline 
     
          \hline 
  \cellcolor[gray]{0.9}{\scriptsize   \cite{armknecht2014outsourced}}&\multirow{2}{*}{\rotatebox[origin=c]{0}{\scriptsize }} \scriptsize $\times$&\scriptsize$\times$&\scriptsize$\checkmark$\\
     \cline{2-4}
      
      \hline
      
       \hline
      
\cellcolor[gray]{0.9}{\scriptsize   \cite{xu2016lightweight}}&\multirow{2}{*}{\rotatebox[origin=c]{0}{\  \scriptsize }} \scriptsize$\times$&\scriptsize$\times$&\scriptsize$\times$\\
     \cline{2-4}


%%%%%%%%%%%%%%%%%%%%%%%
      \hline
      
       \hline
      
\cellcolor[gray]{0.9}{\scriptsize   \cite{Storage-Time}}&\multirow{2}{*}{\rotatebox[origin=c]{0}{\  \scriptsize }} \scriptsize$\times$&\scriptsize$\checkmark^{*}$&\scriptsize$\checkmark^{*}$\\
     \cline{2-4}


%%%%%%%%%%%%%%%%%%%%%%
 \hline
\end{tabular}
%}
\end{center}
%\end{footnotesize}
\end{table*}


 \noindent\textbf{\textit{Computation Complexity}}. In our analysis, we do not take into account the cost of erasure-coding a file, as it is identical in all schemes. We first analyse the computation cost of  SO-PoR.  A client in step \ref{gen-client-server-tags} performs $n$  multiplications and $n$ additions to generate permanent tags. In step \ref{Gen-Disposable-Tags}, it performs $cz$ multiplications and $cz$ additions to generate disposable tags for $z$ verifications. The client in step \ref{Gen-Puzzles-} invokes $\mathtt{GenPuz(.)}$ function, in C-TLP, that  costs $O(z)$. So, the client's total computation cost of preparing and storing a file is $O(n+cz)$. Now we consider the cloud server's cost that can be categorised into two classes: (a) solving a puzzle: $\mathtt{SolvPuz}()$, run only once, and (b) generating PoR run for each verification. In particular, the cloud in step \ref{Solve-Puzzle-Regen-Indices},  invokes $\mathtt{SolvPuz}()$, in C-TLP, that costs $O(T z)$ this includes the cost in step \ref{fully-recover-l} as well. To compute proofs, in step \ref{Gen-PoR}, it performs $2 c z$ multiplications and  $2 c z$ additions. So, the server to generate proof   is $O(cz)$. Next, we analyse the cost of the smart contract. In step \ref{check-hash}, it invokes $\mathtt{Verify}(.)$, in C-TLP, that in total costs $O(z)$, this involves invoking $z$  hash function's instances. Also, the contract in step \ref{verify-PoR} performs $z(1+c)$ and $z(1+c)$ modular multiplications and additions respectively. Thus, the total cost  is $O(cz)$ involving  mainly modular additions and multiplications.
%\cite{armknecht2014outsourced,xu2016lightweight}.

Now we analyse the computation cost of \cite{armknecht2014outsourced}. To prepare file tags, a client performs: $n$ multiplications and $n$ additions. Also, to verify tags generated by the auditor, the client  has to engage in a zero-knowledge protocol that requires it to carry out   $6n$ exponentiations and $2 n$ multiplications. Therefore, the client's computation complexity is $O(n)$. Also, the auditor in total performs $3n$ multiplications, $3 n$ additions and $3 n$ exponentiations to prepare file's metadata, so its complexity at this phase is $O(n)$. For the cloud to generate $z$ proofs, in total it performs $2z(c+c')$ multiplications and the same number of additions, where $c'$ is the number of challenges sent by the auditor to the cloud (on client's behalf) and $c>c'$, e.g. $c'=(0.1)c$. So, the total complexity of the cloud is $O(z  (c+c'))$. Next, we consider the verification cost. The auditor performs $z(1+c) $ multiplications and $z(1+c) $ additions to verify PoR. It also  performs $2cz $ additions in \textit{CheckLog} algorithm that requires the client to perform $z(2c+1)$ additions and $cz$ multiplications. Nevertheless, as discussed in Section \ref{Related-Work}, running only \textit{CheckLog} does not allow the client to detect a misbehaving auditor. Thus, it has to run \textit{ProveLog} too, that requires the client to perform $cz$ multiplications and $cz$ additions and requires the auditor to reveal all its secrets to the client. So, the total verification complexity is $O(cz)$. Now we turn our attention to the computation complexity of \cite{xu2016lightweight}. To prepare metadata, the client needs to perform $2  n$ multiplication and $2  n$ additions, so the client's complexity is $O(n)$. For the cloud to generate a proof it needs to perform $3  z$ exponentiations,  $z  (3  c+6)$ multiplications  and $z  (3 c+2)$ additions. So, its complexity is $O(cz)$. On the other hand, the verifier performs $cz$ exponentiations to compute  challenges. To verify the proof, it carries out  $6 z$ exponentiations, $cz$ multiplications,    $cz$ additions, and  $7  z$ pairings. Therefore, the verifier complexity is  $O(cz)$ dominated by expensive exponentiations and pairing operations. Also, we analyse the  computation cost of efficient privately verifiable PoR  in \cite{DBLP:conf/asiacrypt/ShachamW08}. A client in the store phase performs $n$ multiplications and $n$ additions to construct the tags. So, its complexity is $O(n)$. A server performs $2cz$ multiplications and $2cz$ additions to generate proofs, so in total $4cz$ or $O(cz)$ modular operations for $z$ verifications it carries out. The client, as verifier this time, performs in total $2z(1+c)$ or $O(z(1+c))$ modular operations.  Next, we turn our attention to  \cite{Storage-Time}, and evaluate the cost of the protocol that supports public verifiability, basic PoSt. As stated by the authors, they add a VDF to  the PoR construction in \cite{Filecoin} which uses a Merkle tree-based PoR. Since,  this PoR scheme only involves invocations of hash function, here we only focus on VDF cost as it dominates other computation costs. We assume that the most efficient    publicly verifiable delay function VDF \cite{Wesolowski19} is used.  In the setup, the client constructs a Merkle tree on the entire data  by invoking a hash function many times and also generates a random challenge. We ignore these costs as they are dominated by VDF's costs. To generate $z$ PoR's, the server invokes VDF $z$ times that in total runs in time period $T$. This involves $3Tz$ modular exponentiations over $\mathbb{Z}_{\scriptscriptstyle N}$ (where $N$ is a RSA modulus), and $Tz$ modular multiplications. Therefore, its complexity is $O(Tz)$. For a validator to ckeck the proofs output by VDF, it performs $3z$ modular exponentiations over $\mathbb{Z}^{*}_{\scriptscriptstyle N}$ (or $\bmod\phi(N)$). So the verifier's complexity is $O(z)$. 



Now we compare the protocols above. The verification in SO-PoR is much faster than the other three protocols; firstly, it requires no exponentiations in this phase, whereas \cite{xu2016lightweight,Storage-Time} do, and secondly, it requires $\frac{9c+3}{2(1+c)}$ times fewer computation than \cite{armknecht2014outsourced};  Specifically, when $c=460$,  SO-PoR verification requires about $4.5$ times fewer computation than the verification in \cite{armknecht2014outsourced} needs. In SO-PoR, the cloud server needs to perform $Tz$ exponentiations to solve puzzles, however this is independent of the file size. The sever in \cite{Storage-Time} also performs $3Tz$ modular exponentiations, which is  $3$ times higher than the number of exponentiations done by the server in SO-PoR. The  protocols in \cite{armknecht2014outsourced,DBLP:conf/asiacrypt/ShachamW08} do not include the puzzle-solving procedure (and they do not offer all features that SO-PoR does).  Furthermore, the proving cost in SO-PoR is similar to that of in \cite{armknecht2014outsourced}, and is  much better than \cite{xu2016lightweight}, as the latter one requires both exponentiations and pairing operations while the prove algorithm in SO-PoR  does not involve any exponentiations. The  proving cost in \cite{Storage-Time} is the lowest, as it requires only invocations of a hash function.  Also, the store phase in SO-PoR has a much lower computation cost than the one in \cite{armknecht2014outsourced}. The reason is that the number of exponentiations required (in this phase) in SO-PoR is independent of file size and is only linear with the number of delegated verifications; however, the number of exponentiations in \cite{armknecht2014outsourced} is linear with the file size. For instance, when $||{\bm{F}}||=1$-GB, the  total number of blocks is:   $n=\frac{1-\text{GB}}{128-\text{bit}}=625\times 10^{\scriptscriptstyle 5}$. Since the number of exponentiations in \cite{armknecht2014outsourced} is linear with the number of blocks, i.e. $9n$, the total number of exponentiations imposed by store algorithm is: $5625\times 10^{\scriptscriptstyle 5}$ which is very high. This is the reason why in the experiment in \cite{armknecht2014outsourced} only a small file size: $64$-MB, is used, that can be stored locally without the need to use  cloud storage, in the first place.   Now, we turn our attention to SO-PoR. Let the verification be done every month for a $10$-year period, in this case,  $z=120$.  So, the total number of exponentiations required by the store in SO-PoR is $121$. This means the store phase in SO-PoR requires over $46\times 10^{\scriptscriptstyle 5}$ times  fewer exponentiations than the one in \cite{armknecht2014outsourced} needs. On the other hand, the store algorithm in \cite{xu2016lightweight} does not involve any exponentiations; however, its  number of modular additions and multiplication  is higher than the ones imposed by the store in SO-PoR. The  store cost in \cite{Storage-Time} is the lowest, as it requires only invocations of a hash function. Furthermore, the verification and prove cost of SO-PoR and privately verifiable PoR \cite{DBLP:conf/asiacrypt/ShachamW08} are identical. 




%\noindent\textbf{\textit{I/O Cost}}. In SO-PoR and \cite{xu2016lightweight,armknecht2014outsourced}, for a server to generate a PoR, it only needs to access a constant number of blocks. Therefore, their total I/O cost is $O(z)$. But, unlike the majority of existing PoR schemes, in \cite{Storage-Time}  the server has to access the \emph{entire file blocks} to generate a PoR, so its cost is much higher than the rest. In particular, its total I/O cost is $O(|F|z)$. Note that I/O cost plays a crucial role in the scalability of the server and its ability to serve multiple clients/queries degrades when the I/O cost is significantly high. 




%Note, the total number of exponentiations required to prepare a file is $9\cdot n$. 

%Therefore, after running  \textit{ProveLog}, the auditor (who may not trust the client) has to   run again store algorithm which requires: (a) $9\cdot n$ exponentiations, and (b) downloading the entire file. 
 
  
% 
%  \noindent\textbf{\textit{I/O Cost}}. 
%  
%In SO-PoR and�\cite{armknecht2014outsourced,DBLP:conf/asiacrypt/ShachamW08}, for a server to generate a PoR, it only needs to access a constant number of blocks.
%  
%  
%  Therefore, their I/O cost in total is $O(z)$. However, in \cite{Storage-Time} the server has to access the entire file blocks to generate a proof, therefore its total I/O complexity is much higher, i.e. $O(|F|z)$.
% 
% 
 
 
 \
 
  \noindent\textbf{\textit{Communication Complexity}}. In our analysis, we do not take into account the communication cost of uploading an encoded file, i.e. $||{\bm{F}}||$, when the client for the first time sends it to the cloud, as it is identical in all schemes. The communication cost of SO-PoR is as follows. The client, in step \ref{Outsource-File}, sends $n$ permanent tags, $z c$ disposable tags, and the output of $\mathtt{GenPuz}(.)$ to the server and contract, where each (permanent/disposable) tag: $\sigma_{\scriptscriptstyle j}\in \mathbb{F}_p$ and $|\sigma_{\scriptscriptstyle j}|=128$-bit. Note the client also sends a few public parameters: $\hat{pk}$, whose size is short.  Therefore, the client's bandwidth  is: $128  (n+ c  z+19 z)$ bits, while its communication complexity is $O(n+c z)$. The cloud in step \ref{Register-Proofs}, sends $z$ pairs $(\mu_{\scriptscriptstyle j},\xi_{\scriptscriptstyle j})$, where $\mu_{\scriptscriptstyle j},\xi_{\scriptscriptstyle j}\in \mathbb{F}_p$ and $|\mu_{\scriptscriptstyle j}|=|\xi_{\scriptscriptstyle j}|=128$-bit. Also, in step \ref{fully-recover-l}, it sends to the contract the output of $\mathtt{Prove}(.)$, in  C-TLP, whose total size is $628 z$ bits. So, the clouds total bandwidth is about $884  z$ bits and its complexity is $O(z)$ which is independent of and constant in the file size. 
 
 The communication cost of \cite{armknecht2014outsourced} is as follows. The client sends $n$ tags to the server, where the size of each tag is about $128$ bits. So its bandwidth is $128  n$, and its complexity is  $O(n)$. The auditor also sends $n$ tags to the server, where each tag size is also $128$ bits. It also sends the tags to the client along with $zk$ proofs that contain $4n$ elements in total,  where $2n$ of them are elements of $\mathbb{Z}_{\scriptscriptstyle N}$ and each element size is  $2048$ bits, and each of the other $2n$ elements is $160$ bits long. Also, the auditor in \textit{ProveLog} sends $z$ pairs to the client with the  bandwidth of $256  z$. So, the auditor's total bandwidth and complexity is  $4672  n+256  z$ and $O(n+z)$ respectively. Moreover, the cloud sends the entire file, $F$, to the auditor in the store phase and also sends $2  z$ pairs of PoR to the auditor, where each element of the pair is of size $128$ bits. Therefore, the cloud's total  bandwidth is $||{\bm{F}}||+256  z$, while its complexity is $O(||{\bm{F}}||+z)$. Note that the cloud's proof size complexity is constant and independent of the file size, i.e. $O(1)$. Now, we analyse the communication cost of  \cite{xu2016lightweight}. The client bandwidth and complexity are $2048  n$ and $O(n)$ respectively, as it sends to the cloud $2  n$ tags, where each tag size is $1024$ bits. Also, the cloud bandwidth  and complexity are $6144 z$ and $O(z)$ respectively, as for each verification the cloud sends to the verifier $6$ elements each of them is $1024$-bit long. Furthermore, in \cite{DBLP:conf/asiacrypt/ShachamW08} the client bandwidth in the store phase is $128n$, while the server  bandwidth is $256z$. In this scheme, the complexity of a proof size is $O(1)$. In \cite{Storage-Time}, the client can send only the file and random challenge of size $128$-bit to the server, who creates a Merkle tree on top of the file's blocks. Therefore, the client's bandwidth is  $128$-bit. The server sends to the verifier $cz$ PoR proofs that cost it in total  at least $128cz\log(n)$ bits. Also, the server sends VDF's proofs for $z$ outputs, that cost in total $4096z$.  Therefore, the server's total bandwidth is $(128cz\log n)+4096z$. The reason the cost involves $c$ (the number of challenges) is that unlike the other three schemes, this scheme does not support a linear combination of tags/proofs (e.g. homomorphic tags), and in each proving phase $c$ proofs are generated.  Furthermore, in this scheme, the complexity of a proof size is logarithmic with the number of file blocks,  $O(\log(n))$.
 
 To conclude, the verifier-side bandwidth of SO-PoR (and \cite{xu2016lightweight,Storage-Time}) is much lower than \cite{armknecht2014outsourced}. For instance, when $||{\bm{F}}||=1$-GB and $z=100$, a verifier in SO-PoR requires $62\times 10^{\scriptscriptstyle 6}$  fewer bits than the one in \cite{armknecht2014outsourced} does. A client in SO-PoR has a higher bandwidth than it would have in the rest of the protocols. But, this cost is one-off, at the setup phase.  The server-side bandwidth of SO-PoR is the lowest;  for instance (for the same parameters above) a server in SO-PoR requires $9\times 10^{\scriptscriptstyle4}$,  $7$, and $1729$ times fewer bits  than those required in \cite{armknecht2014outsourced}, \cite{xu2016lightweight} and \cite{Storage-Time} respectively.   Moreover, \cite{Storage-Time} has the worst proof size complexity, which is logarithmic to the file size; while the proof size complexity of the rest of the schemes  is constant.  Thus, SO-PoR's server-side bandwidth is significantly lower than the rest  while having constant proof size. 
 
 
 
 

 
 
 
 \begin{remark}
In \cite{armknecht2014outsourced}, the  additional costs   to secure parties against a malicious client stem from only the store phase, where  an auditor downloads the entire file, generates zero-knowledge proofs, and has the client sign them after verifying the proofs. Therefore, the overheads of proving and verifying phases, in this protocol, would remain unchanged if the protocol considers an honest client.  
 \end{remark}











%In general, the overall bandwidth of SO-PoR is much lower than \cite{armknecht2014outsourced}, and is about $9\times$ higher than the outsourced PoR that requires a \emph{trusted} verifier, i.e. \cite{xu2016lightweight}, due to  higher  client-side bandwidth.  Also, a client bandwidth   in SO-PoR requires $128(cz+19z)$  more bits than a client in the privately verifiable PoR \cite{DBLP:conf/asiacrypt/ShachamW08}, while the server's bandwidth in SO-PoR is $3.4$ times higher than that in \cite{DBLP:conf/asiacrypt/ShachamW08}.



