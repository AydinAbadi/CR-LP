% !TEX root =main.tex

\section{Conclusion}


Time-lock puzzles are important cryptographic protocols with various applications. Nevertheless,  existing puzzle schemes are not suitable to deal with multiple puzzles at once.  In this work, we put forth the concept of composing multiple puzzles, where given puzzles composition at once, a server can find one puzzle's solution after another. This process does not require the server to deal with all of them in parallel which reliefs the server from having numerous parallel processors and allows it to save considerable computation overhead. We proposed a candidate construction: chained  time-lock puzzle (C-TLP) that possesses the aforementioned features. Furthermore, C-TLP is equipped with an efficient verification algorithm publicly executable.  We also illustrated how to use C-TLP to construct an  efficient outsourced  proofs of retrievability scheme that supports \emph{real-time detection} and \emph{fair payment} while keeping its costs considerably lower than the state of the art protocols. Moreover, we showed how VDF's in certain settings can be replaced with C-TLP to gain considerable cost improvement.






%In the future, we would like to investigate how multiple puzzles originated from \emph{distinct clients} can be composed  while offering the same features as C-TLP does. Exploring other applications of C-TLP would be another interesting future research direction. 


