% !TEX root =main.tex


\vspace{-7mm}

\section{Smarter Outsourced PoR (SO-PoR) Utilising C-TLP}
   %\vspace{-5mm}
 As discussed in Section \ref{Related-Work}, the existing outsourced PoR's  have serious shortcomings, e.g. having high costs, not supporting real-time detection, or suffering from the lack of a fair payment mechanism. In this section, we present SO-PoR to addresses them. 
 
 

 \vspace{-3mm}
 
\subsection{SO-PoR Overview} 

%SO-PoR uses a unique combination of (a) homomorphic MAC-based PoR \cite{DBLP:conf/asiacrypt/ShachamW08}, (b) C-TLP, and (c) a smart contract. It uses the MAC-based PoR, due to its high efficiency. Since the MAC's are privately verifiable and secret verification keys are needed to check PoR proofs, it also uses C-TLP to efficiently make them publicly verifiable. In this case, C-TLP encapsulates the verification keys and reveals each of them to verifiers only after a certain time. This combination allows the protocol to \emph{take advantage of MAC's efficiency in the setting where public verifiability is needed}. The combination of the two primitives has applications beyond PoR. SO-PoR also utilises a smart contract who acts as a public verifier on the client's behalf to verify proofs and pay an honest server. The MAC's and CTL combination also makes it possible to use a smart contract (which does not inherently support a private state) while keeping costs very low. 

SO-PoR uses a unique combination of (a) homomorphic MAC-based PoR \cite{DBLP:conf/asiacrypt/ShachamW08}, (b) C-TLP,  (c) a smart contract,  (d) a pre-computation technique, and (e) blockchain-based random extraction beacon \cite{DBLP:journals/iacr/AbadiCKZ19,armknecht2014outsourced}. It uses the MAC-based PoR, due to its high efficiency. Since the MAC's are privately verifiable and secret verification keys are needed to check PoR proofs, SO-PoR also uses C-TLP to efficiently make them publicly verifiable. In this case, C-TLP encapsulates the verification keys and reveals each of them to verifiers only after a certain time. SO-PoR also utilises a smart contract which acts as a public verifier on the client's behalf to verify proofs and pay an honest server. The pre-computation technique allows the client at setup to generate a constant number of \emph{disposable} homomorphic MAC's for each verification.  The combination of disposable homomorphic MAC's and C-TLP  makes it possible to (a) use a smart contract  and (b) take advantage of MAC's efficiency in the setting where public verifiability is needed. This combination has applications beyond PoR.  A blockchain-based random extraction beacon allows the server to independently  derive a set of unpredictable random values from the blockchain such that the values' correctness is publicly verifiable. 



At a high-level  SO-PoR works as follows. The client encodes its file using an error-correcting code and  for each $j\text{\small{-th}}$ verification it does the following. It picks two   random keys: $(v_{\scriptscriptstyle j},l_{\scriptscriptstyle j})$ of a $\mathtt{PRF}$. It uses $v_{\scriptscriptstyle j}$ to generate $c$ random blocks' indices, i.e. challenged blocks. It utilises  $l_{\scriptscriptstyle j}$ to generate a disposable MAC on each challenged block. It also uses C-TLP to make two puzzles, one that encapsulates $v_{\scriptscriptstyle j}$, and  another that encapsulates $l_{\scriptscriptstyle j}$. It deposits enough coins to cover $z$ successful PoR verifications in a smart contract. The client sends the encoded file, tags and the puzzles to the server. When $j\text{\small{-th}}$ PoR proof is needed, the server manages to discover key $v_{\scriptscriptstyle j}$ that lets it determine which file blocks are challenges. The server also uses the beacon to extract a set of random values from the blockchain. Using the MAC's,  challenged blocks, and  beacon's outputs, the server generates a compact PoR proof. The server sends the proof to the contract. After that, it can delete the related disposable MAC's.  For the same verification, after a fixed time, it manages to find the related MAC's verification key: $l_{\scriptscriptstyle j}$. It sends the key to the server who checks the correctness of $l_{\scriptscriptstyle j}$ and  PoR proof. If the contract accepts all proofs, then it pays the server  for $j\text{\small{-th}}$ verification; otherwise, it notifies the client.  







\vspace{-3mm}
\subsection {SO-PoR Model Overview}\label{SO-PoR-Model}
SO-PoR model is built upon the traditional PoR paradigm \cite{DBLP:conf/asiacrypt/ShachamW08} which   is a challenge-response  protocol where a server proves to  an honest client that its file is retrievable (see  Appendix \ref{PoR-Model} for a formal definition of the PoR). In  SO-PoR, however,  a client may not be available for   verification. So, it wants to  delegate  a set of verifications that it cannot carry out. Informally, in this setting, it (in addition to file retrievability)  must have three guarantees: (a) \emph{verification correctness}: every verification is performed honestly, so  the client can trust the verification's result  without redoing it, (b) \emph{real-time detection}: the client is notified in almost real-time when a  proof is rejected, and (c) \emph{fair payment}: in every verification, the server is paid only if a  proof  is accepted. In SO-PoR, three parties are involved: an honest client, potentially malicious server  and a standard smart contract. SO-PoR also, analogous to  \cite{DBLP:conf/asiacrypt/ShachamW08},  allows a client to perform the verification itself,  when it is available. We present our formal definition of SO-PoR in  Appendix \ref{SO-PoR-Model}.  

\vspace{-3mm}
%SO-PoR uses a novel combination of disposable tags, pre-computation technique,  C-TLP scheme, smart contract,  pseudorandom functions and commitment scheme.  At a high-level, SO-PoR works as follows. The client generates a set of authenticator tags on the file. These tags will allow the client to verify its data availability  when it is online. Also, for every verification that the client cannot be online, it precomputes a (small) set of \emph{disposable} tags related to the file's blocks that will be challenged for that verification. However, unlike standard PoR schemes in which   a subset of file blocks are challenged by picking their indices  randomly on the fly just before the verification, in SO-PoR, the challenged blocks' indices are picked \emph{pseudorandomly} by the client in the setup phase. Then, the client for, each verification, encodes the secret key that allows regeneration of the pseudorandom indices and a secret key used for a PoR verification into two puzzles. The client stores the file, tags, and puzzles on the server. It stores  commitments of the secret values that will be used for PoR verification  in a smart contract. Also, it deposits enough coins in the smart contract, to pay the server if each proof (given by the server) is accepted. At each verification time, the server first solves a puzzle and fully recovers the key for random indices. Using the key,  corresponding tags and the file, it generates a PoR and sends it to the contract. After a certain  period, for the same verification, it manages to fully find another puzzle solution  which is the verification key. It sends the key to the contract  who first checks the correctness of the key and then verifies the PoR. If accepted, the contract  for that verification pays the cloud server  who can now delete all metadata (e.g. tags, encrypted, and decrypted values) for that verification.  



%First, client breaks up its file into blocks and apply an error-correcting code on every blocks. Then, it generates a set of MAC-based tags on every  blocks. These tags will allow the client to verify its data availability when it is online. For the sake of simplicity, let us assume the client does not want to perform  $z$ consecutive verifications. For  every $j^{\scriptscriptstyle th}$ verification  ($1\leq j\leq z$)  the client determines the random indices of the blocks that will be challenged for this verification and also precomputes (small) set of MAC-based tags for those blocks. It uses time-lock encryption scheme to encrypts the random indices and secret verification values (for the tags) and stores the encrypted values on the server. It also  stores the hash of random indices and  secret verification values in a smart contract. The client also stores the indices of the blocks that will appear in the blockchain from which a set of random value will be extracted. 

%% !TEX root =main.tex




\section {SO-PoR Model}\label{SO-PoR-Model}
In this section, we provide a formal definition  of SO-PoR. As previously stated, it builds upon the traditional PoR model \cite{DBLP:conf/asiacrypt/ShachamW08}, presented in Appendix \ref{PoR-Model}.  In  SO-PoR, unlike the traditional PoR, a client may not be available every time  verification is needed. Therefore, it wants to  delegate  a set of verifications that it cannot carry out itself. In this setting, it (in addition to file retrievability)  must have three guarantees: (a) \emph{verification correctness}: every verification is performed honestly, so  the client can rely on the verification result  without the need to re-do it, (b) \emph{real-time detection}: the client is notified in almost real-time when server's  proof is rejected, and (c) \emph{fair payment}: in every verification, the server is paid only if the server's  proof  is accepted. In SO-PoR, three parties are involved: an honest client, potentially malicious server  and a standard smart contract. SO-PoR also allows a client to perform the verification itself, analogous to the traditional  PoR, when it is available. 

%To satisfy the aforementioned requirements, and keep verifications' cost low, SO-PoR mainly utilises a smart contract (for verification and payment) and the chained time-lock puzzle to eventually release secret values used to: (a) generate challenges and (b) verify proofs. Therefore, i


\begin{definition}
A Smart Outsourced PoR (SO-PoR) scheme consists of seven algorithms ($\mathtt{Setup}, \mathtt{Store},$ $ \mathtt {SolvPuz}, $ $ \mathtt{GenChall}, \mathtt{Prove},$ $ \mathtt{Verify},  \mathtt{Pay}$) defined below: 


\
\begin{itemize}
\item[$\bullet$] $\mathtt{Setup}(1^{\scriptscriptstyle\lambda},\Delta, z)\rightarrow (\hat{sk},\hat{pk})$:  a probabilistic algorithm, run by a client.  It  takes as input a security: $1^{\scriptscriptstyle\lambda}$, time parameter: $\Delta$, and the number of verification delegated: $z$. It  outputs a set of  secret and public keys.

\

\item[$\bullet$] $\mathtt{Store}(\hat{sk},\hat{pk}, F,z)\rightarrow ({\bm{F}}, \sigma, \vv{\bm{o}},aux)$: a probabilistic algorithm, run only once by a client. It  takes as  input the secret key: $\hat{sk}$, public key: $\hat{pk}$, a file: $F$, and the number of verifications: $z$ that the client wants to delegate. It outputs an encoded file: ${\bm{F}}$,  a set of tags: $\sigma$, a set of $z$ puzzles: $\vv{\bm{o}}$, and public auxiliary data: $aux$. First three outputs are stored on the server and last output: $aux$, is   stored on a smart contract. 

\

\item[$\bullet$] $\mathtt {SolvPuz}(\hat{pk},\vv{\bm{o}})\rightarrow \vv{\bm{s}}$:  a deterministic algorithm that takes as input the public key: $\hat{pk}$ and puzzle vector: $\vv{\bm{o}}$.  It for each  $j\text{\small{-th}}$ verification outputs a  pair: $\ddot{s}_{\scriptscriptstyle j}:(v_{\scriptscriptstyle j},l_{\scriptscriptstyle j})$ of solutions, where $v_{\scriptscriptstyle j}$ and $l_{\scriptscriptstyle j}$ are outputted at time $t_{\scriptscriptstyle j}$ and $t'_{\scriptscriptstyle j}$ respectively and $t'_{\scriptscriptstyle j}> t_{\scriptscriptstyle j}$. Therefore, the algorithm in total outputs $z$ pairs. Value $l_{\scriptscriptstyle j}$ is sent  to the smart contract right after it is discovered. This algorithm is run  by the server.

\


\item[$\bullet$] $\mathtt{GenChall}(j,|{\bm{F}}|, 1^{\scriptscriptstyle\lambda},\ddot{s}_{\scriptscriptstyle j},aux)\rightarrow \vv{\bm{c}}$: a probabilistic algorithm that takes as input a verification index: $j$, the encoded file size: $|{\bm{F}}|$, security parameter: $1^{\scriptscriptstyle\lambda}$, first component of the related solution pair, $v_{\scriptscriptstyle j}\in \ddot{s}_{\scriptscriptstyle j}$, and public parameters: $pp\in aux$ containing  a blockchain and its parameters. It outputs pairs $\ddot{c}_{\scriptscriptstyle j} : (x_{\scriptscriptstyle j} , y_{\scriptscriptstyle j} )$, where each pair includes a pseudorandom  block's index:  $x_{\scriptscriptstyle j}$ and random coefficient: $y_{\scriptscriptstyle j}$. Also, values $x_{\scriptscriptstyle j}$ are derived from $v_{\scriptscriptstyle j}$ while $y_{\scriptscriptstyle j}$ are derived from $pp$. This algorithm is run by the server for each verification. 


%$pp$ is a public parameters for the beacon and it includes, blockchain, chain quality, and index. $\mathtt{GenCoeffs}()$ is called here

\

\item[$\bullet$] $\mathtt{Prove}(j,{\bm{F}}, \sigma,  \vv{\bm{c}})\rightarrow \pi$: a probabilistic algorithm that takes the verification index $j$, encoded file: ${\bm{F}}$ , (a subset of) tags: $\sigma$, and a vector of unpredictable challenges: $\vv{\bm{c}}$, as inputs and outputs a proof of  file retrievability. It is run by the server for each verification.

\

\item[$\bullet$] $\mathtt{Verify}(j,\pi,\ddot{s}_{\scriptscriptstyle j},aux)\rightarrow d:\{0,1\}$: a deterministic algorithm that takes the verification index $j$, proof: $\pi$,  second component of the related solution pair: $l_{\scriptscriptstyle j}\in \ddot{s}_{\scriptscriptstyle j}$, and public auxiliary data: $aux$.  If the proof is accepted, it outputs $d=1$; otherwise, outputs $d=0$. The default value of $d$ is $0$. This algorithm is run by the smart contract for each verification and invoked only once for each verification by only the server. 

\

\item[$\bullet$] $\mathtt{Pay}(j,d)\rightarrow d'=\{0,1\}$: a deterministic algorithm that takes the verification index $j$, the verification output: $d$. If $d=1$, it transfers $e$ amounts to the server and outputs $1$. Otherwise, it does not transfer anything, and outputs $0$. The default value of $d'$ is $0$. The algorithm is run by the  contract, and  invoked only by $\mathtt{Verify}(.)$. 
\end{itemize}
\end{definition}





%
%note that in the above, $\mathtt{Store}$ is a wrapper function that calls $\mathtt{GenPuz}(\vv{\bm{m}},\hat{sk},\hat{pk})$ and $\mathtt{Store}(\hat{sk},F)$ as subroutine,{\color{blue}xx explain what $\vv{\bm{m}}$ is for}
%




  
A SO-PoR scheme must satisfy two main properties: \emph{correctness} and \emph{soundness}. The correctness requires, for any: file, public-private key pairs, and puzzle solutions, both the verification  and pay algorithms, i.e. $\mathtt{Verify}(.)$ and $\mathtt{Pay}(.)$, output $1$ when interacting with  the  prover, verifier, and client  all of which are honest.  The soundness however is split into four properties: extractability, verification correctness, real-time detection, and fair payment, formally defined below.  Before we define the first property,  extractability, we provide the following  experiment between an environment: $\mathcal{E}$ and  adversary: $\mathcal{A}$ who corrupts $C\subsetneq\{\mathcal{S},\mathcal{M}_{\scriptscriptstyle 1},...,\mathcal{M}_{\scriptscriptstyle\beta}\}$, where $\beta$ is the maximum number of miners which can be corrupted in a secure blockchain. In this game, $\mathcal{A}$ plays the role of corrupt parties and $\mathcal{E}$ simulating an honest party's role. 


\begin{enumerate}
\item $\mathcal{E}$ executes $\mathtt{Setup}(.)$ algorithm and provides public key: $\hat{pk}$, to $\mathcal{A}$.   
\item $\mathcal{A}$ can pick  arbitrary file $F'$, and  uses it to make queries to  $\mathcal{E}$ to run:  $\mathtt{Store}(\hat{sk},\hat{pk},$ $ F',z)$ $\rightarrow (F'^{\scriptscriptstyle *}, \sigma, \vv{\bm{o}},aux)$  and return the output to $\mathcal{A}$. Also, upon receiving the output of $\mathtt{Store}()$, $\mathcal{A}$ can locally run  algorithms: $\mathtt {SolvPuz}(\hat{pk},\vv{\bm{o}})$ and   $\mathtt{GenChall}(j,$ $|F'^{\scriptscriptstyle *}|, $ $ 1^{\scriptscriptstyle\lambda},\ddot{s}_{\scriptscriptstyle j},aux)\rightarrow \vv{\bm{c}}$ as well as  $\mathtt{Prove}(j,F^{\scriptscriptstyle *}, \sigma, $ $ \vv{\bm{c}})\rightarrow \pi$,  to get their outputs as well. 
\item $\mathcal{A}$ can request $\mathcal{E}$ the execution of $\mathtt{Verify}(j,\pi,\ddot{s}_{\scriptscriptstyle j},aux)$ for any $F'$ used to query $\mathtt{Store}()$. Accordingly, $\mathcal{E}$ informs  $\mathcal{A}$ about the verification output. The adversary can send a polynomial number of queries to $\mathcal{E}$. Finally, $\mathcal{A}$ outputs the description of a prover: $\mathcal{A}'$ for any file it has already chosen above. 
\end{enumerate}

It is said a cheating prover: $\mathcal{A}'$ is $\epsilon$-admissible if it convincingly answers $\epsilon$ fraction of verification challenges \cite{DBLP:conf/asiacrypt/ShachamW08}. Informally, a SO-PoR scheme supports extractability, if there is an extractor algorithm: $\mathtt{Ext}(\hat{sk},\hat{pk},\mathtt{P}')$, that takes the secret-public keys and the description of the  machine implementing the prover's role: $\mathcal{A}'$ and outputs the file: $F'$. The extractor can reset the adversary to the beginning of the challenge phase and repeat this step polynomially many times for  of extraction, i.e. the extractor can rewind it.

\begin{definition}[$\epsilon$-extractable]\label{extractable} A SO-PoR scheme is $\epsilon$-extractable if  for every adversary: $\mathcal{A}$ who corrupts $C\subsetneq\{\mathcal{S},\mathcal{M}_{\scriptscriptstyle 1} $ $,..., \mathcal{M}_{\scriptscriptstyle\beta}\}$, plays the experiment above, and outputs an $\epsilon$-admissible cheating prover: $\mathcal{A}'$ for a file $F'$,  there exists an extraction algorithm that recovers $F'$ from $\mathcal{A}'$, given honest parties public-private keys and $\mathcal{A}'$,  i.e. $\mathtt{Ext}(\hat{sk},\hat{pk},\mathcal{A}')\rightarrow F'$, except with a negligible probability. 
\end{definition}

% . The extractor has the ability to reset the adversary to the beginning of the challenge phase and repeat this step polynomially many times for the purpose of extraction

In the above game, the environment, acting on honest parties' behalf, performs the verification correctly; which is not always the case in SO-PoR. As the verification can be run by miners a subset of which are potentially corrupted. Even in this case, the verification correctness must hold, e.g.  if a corrupt server sends an  invalid proof then even if $\beta-1$ miners are corrupt (and colluding with it) the verification function will not output $1$ and if the server is honest and submits a valid proof then the verification function does not output $0$ even if $\beta$ miners are corrupt, except with a negligible probability. This is formalised below. 


\begin{definition}[Verification Correctness]\label{Verification-Correctness} Let $\beta$ be the maximum number of miners that can be corrupted in a secure blockchain network and $\lambda'$ be the blockchain security parameter. Also, let $\mathcal{A}$ be the adversary who (plays the above game and) corrupts parties in either $C\subseteq\{\mathcal{S},\mathcal{M}_{\scriptscriptstyle 1},...,\mathcal{M}_{\scriptscriptstyle\beta-1}\}$ or $C'\subseteq\{\mathcal{M}_{\scriptscriptstyle 1},...,\mathcal{M}_{\scriptscriptstyle\beta}\}$.  In SO-PoR, we say the correctness of $j\text{\small-th}$ verification  is guaranteed if: 
 
$$\begin{array}{l}
\text{in the former case}: Pr[\mathtt{Verify}_{\scriptscriptstyle C}(j,\pi,\ddot{s}_{\scriptscriptstyle j},aux)=1]\leq \mu(\lambda')\\
\text{in the latter case}: Pr[\mathtt{Verify}_{\scriptscriptstyle C'}(j,\pi,\ddot{s}_{\scriptscriptstyle j},aux)=0]\leq \mu(\lambda')
\end{array}$$
where $\mu(.)$ is a negligible function. 
\end{definition}

Also, a client needs to have a guarantee that for each verification it can get a correct result within a (fixed) time period. 

\begin{definition}[$\Upsilon$-real-time Detection]\label{real-time Detection} Let $\mathcal{A}$, as defined above, be the adversary who corrupts either $C$ or $C'$.
A client, for each $j\text{\small{-th}}$ delegated verification, will get a correct output of  $\mathtt{Verify(.)}$, by  means of reading a blockchain, within time window $\Upsilon$, after the time when the server is supposed to send  its proof  to the blockchain network. Formally,

$$\mathtt{Read}(\Upsilon,\mathtt{Verify}_{\scriptscriptstyle D}(j,\pi,\ddot{s}_{\scriptscriptstyle j},aux))\rightarrow \{0,1\}$$
where $D\subsetneq\{C,C'\}$, except with a negligible probability. 
\end{definition}





\begin{definition}[Fair Payment]\label{Fair-Payment}  SO-PoR supports a fair payment if the client and server fairness are satisfied: 

\begin{itemize}
\item[$\bullet$] \textit{\textbf{Client Fairness}}: An honest client is guaranteed that it only pays ($e$ coins) if the server provides an accepting proof, except with a negligible probability. 
\item[$\bullet$]\textit{\textbf{Server Fairness}}: An honest server is guaranteed that the client gets a correct proof if the client pays ($e$ coins),   except with a negligible probability. 
\end{itemize}
Formally, let $\mathcal{A}$ be the adversary who corrupts either $C$ or $C'$, as defined above. To satisfy a fair payment:
\begin{equation}
Pr[\mathtt{Pay}_{\scriptscriptstyle D}(.)=b 	\cap  \mathtt{Verify}_{\scriptscriptstyle D}(.)=b]\geq 1-\mu(\lambda'),   
\end{equation}

the following inequality must hold:
\begin{equation}\label{inequ::fair-payment}
Pr[\mathtt{Pay}_{\scriptscriptstyle D}(.)=b' 	\cap \mathtt{Verify}_{\scriptscriptstyle D}(.)=b] \leq \mu(\lambda'),
\end{equation}
where $D\subsetneq\{C,C'\},b\neq b'$, and $b, b'\subsetneq\{0,1\}$









%\begin{equation}
%Pr[\mathtt{Pay}_{\scriptscriptstyle D}(.)=1 	\cap \mathtt{Verify}_{\scriptscriptstyle D}(.)=0] \leq \mu(\lambda')
%\end{equation}
%\begin{equation}
%Pr[ \mathtt{Pay}_{\scriptscriptstyle D}(.)=0 	\cap \mathtt{Verify}_{\scriptscriptstyle D}(.)=1] \leq \mu(\lambda')
%\end{equation}
%\begin{equation}
%Pr[\mathtt{Pay}_{\scriptscriptstyle D}(.)=1 	\cap  \mathtt{Verify}_{\scriptscriptstyle D}(.)=1]\geq 1-\mu(\lambda')
%\end{equation}
%where $D\subsetneq\{C,C'\}$
\end{definition}

The above definition also takes into account the fact that the client at the time of delegated verification is not necessarily available to make the payment itself, so the payment is delegated to a third party, e.g. a smart contract. In this case, the definition  ensures that even if  the client or/and server are honest, the third party cannot affect  the fairness (except with a negligible probability).

% Moreover, it is not hard to see, if the inequality \ref{} holds, then the fairness is guaranteed, with a high probability:  \begin{equation*}
%Pr[\mathtt{Pay}_{\scriptscriptstyle D}(.)=b 	\cap  \mathtt{Verify}_{\scriptscriptstyle D}(.)=b]\geq 1-\mu(\lambda')
%\end{equation*} 


%In the following we explain the rational behind the above definition. In SO-PoR scheme, for each verification, the server sells  a proof: $\pi$ to a client and earns  $e$ coins if and only if the proof is accepted, i.e. $\mathtt{Verify}(.)=1$. In SO-PoR setting, the client at the time of delegated verification is not necessarily online to make the payment itself, so it is done by a third party (e.g. a smart contract). The definition must ensure that   even if both server and client are honest, the third party cannot affect  the fairness. 


\begin{definition}[SO-PoR Security]\label{SO-PoR-Security} A SO-PoR scheme is secure if it is $\epsilon$-extractable, and satisfies verification correctness, $\Upsilon$-real-time detection, and fair payment properties.

\end{definition}




\begin{remark}
The folklore assumption is that (in a secure blockchain) a smart contract function \emph{always outputs a correct result}. However, this is not the case and it may fail under certain circumstances.  For instance, as shown in \cite{LuuTKS15} all rational  miners may not verify a certain transaction. As another example,  an adversary (although with a negligibly small probability)  discards a  certain honestly generated blocks,  reverses the state of blockchain and contract, or breaks a client's signature scheme.  Accordingly, in our definitions above, we take such cases  into consideration and allow the possibility that a function outputs an incorrect result even though with a negligibly small probability. 
\end{remark}




\begin{remark}
SO-PoR model differs from traditional (e.g. \cite{DBLP:conf/ccs/JuelsK07,DBLP:conf/asiacrypt/ShachamW08}) and outsourced PoR  (e.g. \cite{armknecht2014outsourced,xu2016lightweight}) models in several aspects. Only  the SO-PoR model offers all the properties. In particular,  traditional PoRs only offer extractability while outsourced ones  offer liability as well, that allows a client (by re-running all verifications function) to detect a verifier if it provides an incorrect verification output, so the client   cannot rely on the verification result provided.  As another difference, the SO-PoR model  takes into account the case where an adversary can corrupt both the server and some miners at the same time.
\end{remark}

\begin{remark}
SO-PoR should also support  the traditional PoR where only client and server interact with each other  (e.g. client generates challenges, and verifies proof) when the client is available. To let SO-PoR definition support that too, we can simply define a flag: $\xi$, in each function, such that  when $\xi=1$, it acts as the traditional PoR; otherwise (when $\xi=0$), it performs as a delegated one. For the sake of simplicity, we let the flag  be implicit in the definitions above, where the default  is  $\xi=1$ 
\end{remark}


\subsection{SO-PoR Protocol}\label{SO-PoR-Protocol}
This section presents SO-PoR protocol in detail, followed by the rationale behind it.      





\begin{enumerate}[leftmargin=.46cm]

\item\textit{\textbf{Client-side Setup}}. 



\begin{enumerate}
%\item Signs and deploys a smart contract: $\mathcal{SC}$ to a blockchain.

% where the contract contains a set of public parameters: e.g. $z$: total number of verifications,  $|F|$: file bit size, $\Delta_{\scriptscriptstyle 1}$:  maximum time period  taken by the server to generate a proof, $\Delta_{\scriptscriptstyle 2}$: time window in which a message is (sent by the server and) received by the contract. The client deposits $e\cdot z$  coins for $z$ successful  verifications. 

\item  \textbf{\textit{\small {Gen. Public and Private Keys}}}:   Picks a fresh key: $\hat{k}$ and two vectors of keys: $\vv{\bm{v}}$ and $\vv{\bm{l}}$, where each vector contains $z$ fresh keys. It picks a large prime number:  $p$ whose size is determined by a security parameter, i.e. $|p|=\iota$.  Moreover, it runs $\mathtt{Setup}(.)$ in  C-TLP scheme to generate a key pair: $(pk, sk)$

\item \textbf{\textit{\small {Gen. Other Public Parameters}}}:  Sets $c$ to the total number of blocks challenged in each verification. It defines  parameters: $w$ and $g$, where  $w$ is an index  of a future block: $\mathcal {B}_{\scriptscriptstyle w}$ in a blockchain that will be added to the blockchain (permanent state) at about the time  first delegated verification will  be done, and $g$ is  a security parameter referring to the number of blocks (in a row) starting from  $w$.  It  sets $z$: the total number of verifications,  $||{\bm{F}}||$: file bit size, $\Delta_{\scriptscriptstyle 1}$:  the maximum time  is taken by the server to generate a proof, $\Delta_{\scriptscriptstyle 2}$: time window in which a message is (sent by the server and) received by the contract, and $e$ amount of coins paid to the server for each successful  verification. Sets $\hat{pk}: (pk,e,g,w,p,c,z,\Delta_{\scriptscriptstyle 1},\Delta_{\scriptscriptstyle 2})$ 


\item\textbf{\textit{\small {Sign and Deploy   Smart Contract}}}: Signs and deploys a smart contract: $\mathcal{SC}$ to a blockchain.  It stores  public parameters: $(z,||{\bm{F}}||, \Delta_{\scriptscriptstyle 1},\Delta_{\scriptscriptstyle 2},c, g,p,w)$, on the contract. It deposits $e z$ coins to the contract. Then, it asks the server to sign the contract. The server signs if it agrees on all parameters.

\end{enumerate}
\item\textit{\textbf{Client-side Store}}.


\begin{enumerate}

\item \textbf{\textit{\small {Encode File}}}: Splits an error-corrected file, e.g. under Reed-Solomon codes, into $n$ blocks; ${\bm{F}}: [F_{\scriptscriptstyle 1},...,F_{\scriptscriptstyle n}]$,  where $ F_{\scriptscriptstyle i}\in \mathbb{F}_p$
\item\label{gen-client-server-tags}\textbf{\textit{\small {Gen. Permanent Tags}}}:  Using the key: $\hat{k}$, it computes $n$ pseudorandom values:  $r_{\scriptscriptstyle i}$ and single value: $\alpha$, as follows.  
 $$\alpha=\mathtt{PRF}(\hat{k},n+1)\bmod p, \  \  \ \  \  \  \forall i, 1\leq i\leq n: r_{\scriptscriptstyle i}=\mathtt{PRF}(\hat{k},i)\bmod p$$
 It uses the pseudorandom values to compute tags for the file blocks. 
 $$\forall i, 1\leq i\leq n: \sigma_{\scriptscriptstyle i}= r_{\scriptscriptstyle i}+ \alpha\cdot F_{\scriptscriptstyle  i}\bmod p$$ 
 So, at the end of this step,  a set of  tags are generated, $\sigma:\{\sigma_{\scriptscriptstyle 1},..., \sigma_{\scriptscriptstyle n}\}$
\item\label{Gen-Disposable-Tags}\textbf{\textit{\small {Gen. Disposable Tags}}}: For   $j\text{\small{-th}}$ verification  ($1\leq j\leq z$):
\begin{enumerate}
\item chooses the related key: $v_{\scriptscriptstyle j}\in\vv{\bm{v}}$ and computes $c$ pseudorandom indices. 
$$\forall b, 1\leq b\leq c: x_{\scriptscriptstyle b,j}=\mathtt{PRF}(v_{\scriptscriptstyle j}, b)\bmod n$$

\item picks the corresponding  key: $l_{\scriptscriptstyle j}\in \vv{\bm{l}}$ and computes $c$ pseudorandom values:  $r_{\scriptscriptstyle b,j}$ and single value: $\alpha_{\scriptscriptstyle j}$, as follows. 
$$\alpha_{\scriptscriptstyle j}=\mathtt{PRF}(l_{\scriptscriptstyle j},c+1)\bmod p, \ \ \ \ \ \forall b, 1\leq b\leq c: r_{\scriptscriptstyle b,j}=\mathtt{PRF}(l_{\scriptscriptstyle j},b)\bmod p$$


\item generates $c$ disposable tags.  
$$\forall b, 1\leq b\leq c: \sigma_{\scriptscriptstyle b,j}=r_{\scriptscriptstyle b,j}+\alpha_{\scriptscriptstyle j}\cdot F_{\scriptscriptstyle y}\bmod p$$
where $y= x_{\scriptscriptstyle b,j}$. At the end of this step, a set $\sigma_{\scriptscriptstyle j}$ of $c$ tags are computed,  $\sigma_{\scriptscriptstyle j}:\{\sigma_{\scriptscriptstyle 1,j},..., \sigma_{\scriptscriptstyle c,j}\}$

%\item encrypts $k_{\scriptscriptstyle 1}^{\scriptscriptstyle j}$ using the time-lock encryption, such that it can be decrypted at time $t^{\scriptscriptstyle j}_{\scriptscriptstyle 1}$, i.e., $C^{\scriptscriptstyle j}_{\scriptscriptstyle 1}=\mathcal{ENC}^{\scriptscriptstyle pk}_{\scriptscriptstyle sk,T}(k_{\scriptscriptstyle 1}^{\scriptscriptstyle j})$, where $T=t^{\scriptscriptstyle j}_{\scriptscriptstyle 1}\cdot S$, and $S$ is a parameter of the encryption: the number of squaring modulo $N$ per second that can be performed by a solver. Also, it uses the time-lock encryption to encrypt $k^{\scriptscriptstyle j}_{\scriptscriptstyle 2}$, such that it can be decrypted at time $ t^{\scriptscriptstyle j}_{\scriptscriptstyle 2}$,  i.e., $C^{\scriptscriptstyle j}_{\scriptscriptstyle 2}=\mathcal{ENC}^{\scriptscriptstyle pk}_{\scriptscriptstyle sk,T'}(k_{\scriptscriptstyle 2}^{\scriptscriptstyle j})$, where $T'= t^{\scriptscriptstyle j}_{\scriptscriptstyle 2}\cdot S$ and $t^{\scriptscriptstyle j}_{\scriptscriptstyle 2}> t^{\scriptscriptstyle j}_{\scriptscriptstyle 1}+\Delta_{\scriptscriptstyle 1}+\Delta_{\scriptscriptstyle 2}$,  $\Delta_{\scriptscriptstyle 1}$  is the time period within which some local computation (by the server) is performed on $k^{\scriptscriptstyle j}_{\scriptscriptstyle 1}$ and $\Delta_{\scriptscriptstyle 2}$ is the time window in which a message, e.g. $k^{\scriptscriptstyle j}_{\scriptscriptstyle 1}$, is (sent by the server and) received by the contract. Note that, the size of time windows are sufficiently large.

%\item\label{gen-hashes} computes $h_{\scriptscriptstyle j}= H(l_{\scriptscriptstyle j})$.


%$\resizeT{\textit{w}}_{\resizeS {\textit  j}}$

%\resizeS {\textit  w}_{\resizeS {\textit  j}}}

\end{enumerate} 

\item\label{Gen-Puzzles-}\textbf{\textit{\small {Gen. Puzzles}}}: Sets $\vv{\bm{m}}=[v_{\scriptscriptstyle 1},l_{\scriptscriptstyle 1},...,v_{\scriptscriptstyle z},l_{\scriptscriptstyle z}]$  and then encrypts the vector's elements, by running: $\mathtt{GenPuz}(\vv{\bm{m}},pk,sk)$ in   C-TLP scheme. This yields a  puzzle vector: $[(V_{\scriptscriptstyle 1},L_{\scriptscriptstyle 1}),...,(V_{\scriptscriptstyle z},L_{\scriptscriptstyle z})]$ and a commitment vector: $\vv{\bm{h}}$. The encryption is done in  such a way that in each $j\text{\small{-th}}$ pair, $V_{\scriptscriptstyle j}$ will be fully decrypted at times $t_{\scriptscriptstyle j}$ and $L_{\scriptscriptstyle j}$ will be decrypted at time $t'_{\scriptscriptstyle j}$, where  $ t_{\scriptscriptstyle j}+\Delta_{\scriptscriptstyle 1}+\Delta_{\scriptscriptstyle 2}\leq t'_{\scriptscriptstyle j} < t_{\scriptscriptstyle j+1}$  %$\Delta_{\scriptscriptstyle 1}$  is the maximum  period  the server needs to generate a proof and $\Delta_{\scriptscriptstyle 2}$ is the time window in which a message is (sent by the server and) received by the contract.




%\item Using  C-TLP scheme, generates a key pair: ($sk_{\scriptscriptstyle 1}, pk_{\scriptscriptstyle 1}$) and  encrypts    $[v_{\scriptscriptstyle 1},...,v_{\scriptscriptstyle z}]$ such that they will be fully decrypted at times $[t_{\scriptscriptstyle 1},...,t_{\scriptscriptstyle z}]$ respectively. This yields a  ciphertext vector: $[V_{\scriptscriptstyle 1},...,V_{\scriptscriptstyle z}]$. Invoking  C-TLP scheme again, it generates key pairs: ($sk_{\scriptscriptstyle 2}, pk_{\scriptscriptstyle 2}$) and  encrypts    $[l_{\scriptscriptstyle 1},...,l_{\scriptscriptstyle z}]$ that will be fully decrypted at times $[t'_{\scriptscriptstyle 1},...t'_{\scriptscriptstyle z}]$ respectively.  This yields a  ciphertext vector: $[L_{\scriptscriptstyle 1},...,L_{\scriptscriptstyle z}]$. Note,   $t'_{\scriptscriptstyle j}\geq t_{\scriptscriptstyle j}+\Delta_{\scriptscriptstyle 1}+\Delta_{\scriptscriptstyle 2}$, where $\Delta_{\scriptscriptstyle 1}$  is the maximum time period  the server needs to generate a proof and $\Delta_{\scriptscriptstyle 2}$ is the time window in which a message is (sent by the server and) received by the contract.

%\item And another one  with input messages  $l_{\scriptscriptstyle 1},...,l_{\scriptscriptstyle z}$ that will be decrypted at times $t'_{\scriptscriptstyle 1},...t'_{\scriptscriptstyle z}$ respectively, where  $t'_{\scriptscriptstyle j}=t_{\scriptscriptstyle j}+\Delta_{\scriptscriptstyle 1}+\Delta_{\scriptscriptstyle 2}$,  $\Delta_{\scriptscriptstyle 1}$  is the time period within which some local computation (by the server) is performed on $v_{\scriptscriptstyle j}$ and $\Delta_{\scriptscriptstyle 2}$ is the time window in which a message, e.g. $v_{\scriptscriptstyle j}$, is (sent by the server and) received by the contract. %Note that, the size of time windows are sufficiently large.



\item\label{Outsource-File}\textbf{\textit{\small {Outsource File}}}: Stores ${\bm{F}},n,\hat{pk}, \{\sigma,\sigma_{\scriptscriptstyle 1},..., \sigma_{\scriptscriptstyle z}, (V_{\scriptscriptstyle 1},L_{\scriptscriptstyle 1}),...,(V_{\scriptscriptstyle z},L_{\scriptscriptstyle z})\}$   on the server. Also, it stores $\vv{\bm{h}}$ on the smart contract. 
\end{enumerate}


\item\textit{\textbf{Cloud-Side Proof Generation}}. For   $j\text{\small{-th}}$ verification  ($1\leq j\leq z$), the cloud:


\begin{enumerate} 
\item\label{Solve-Puzzle-Regen-Indices}\textbf{\textit{\small {Solve Puzzle and Regen.  Indices}}}:   Receives and parses the output of $\mathtt{SolvPuz}(.)$ in C-TLP, to extract $v_{\scriptscriptstyle j}$, at time $t_{\scriptscriptstyle j}$. Using $v_{\scriptscriptstyle j}$, it regenerates $c$ pseudorandom indices. 
$$\forall b, 1\leq b\leq c: x_{\scriptscriptstyle b,j}=\mathtt{PRF}(v_{\scriptscriptstyle j}, b)\bmod n$$ %where $v_{\scriptscriptstyle j}$ is the key fully decrypted by the cloud at time $t_{\scriptscriptstyle j}$ for this verification.


\item \textbf{\textit{\small {Extract Key}}}: Extracts a seed: $u_{\scriptscriptstyle j}$, from the blockchain as follows: $u_{\scriptscriptstyle j}=\mathtt{H}( \mathcal {B}_{\scriptscriptstyle \gamma}||...||  \mathcal {B}_{\scriptscriptstyle \zeta})$, where $\gamma=w+(j-1)\cdot g$ and $\zeta=w+j\cdot g$

\item\label{Gen-PoR}\textbf{\textit{\small {Gen. PoR}}}: Generates a PoR proof.
 $$\mu_{\scriptscriptstyle j}=\sum\limits^{\scriptscriptstyle c}_{\scriptscriptstyle b=1}  \mathtt{PRF}(u_{\scriptscriptstyle j},b)\cdot F_{\scriptscriptstyle y}\bmod p, \  \  \ \xi_{\scriptscriptstyle j}= \sum\limits^{\scriptscriptstyle c}_{\scriptscriptstyle b=1}  \mathtt{PRF}(u_{\scriptscriptstyle j},b)\cdot \sigma_{\scriptscriptstyle b,j}\bmod p$$
 where  $y$ is a pseudorandom index: $y= x_{\scriptscriptstyle b,j}$ %Also, it runs $\mathtt{Prove}(.)$ in C-TLP, to generate a proof: $\ddot{p}_{\scriptscriptstyle j}$, of $v_{\scriptscriptstyle j}$'s correctness
 
 \item\label{Register-Proofs}\textbf{\textit{\small {Register Proofs}}}:  Sends the PoR proof: $(\mu_{\scriptscriptstyle j},\xi_{\scriptscriptstyle j})$   to the smart contract within  $\Delta_{\scriptscriptstyle1}$
 \item\label{fully-recover-l}\textbf{\textit{\small {Solve Puzzle and Regen.  Verification Key}}}: Receives and parses the output of algorithm $\mathtt{SolvPuz}(.)$ in C-TLP to extract $l_{\scriptscriptstyle j}$, at time $t'_{\scriptscriptstyle j}$. Also, it runs $\mathtt{Prove}(.)$ in C-TLP, to generate a proof: $\ddot{p}_{\scriptscriptstyle j}$, of $l_{\scriptscriptstyle j}$'s correctness. It sends $\ddot{p}_{\scriptscriptstyle j}$ (containing $l_{\scriptscriptstyle j}$)  to the contract, so it can be received by the contract within $\Delta_{\scriptscriptstyle 2}$ 
\end{enumerate}


\item \textit{\textbf{Smart Contract-Side Verification}}. For   $j\text{\small{-th}}$ verification  ($1\leq j\leq z$), the contract:



\begin{enumerate} 
\item\textbf{\textit{\small {Check Arrival Time}}}: checks the arrival time of the decrypted values sent by the server. In particular, it checks, if $(\mu_{\scriptscriptstyle j},\xi_{\scriptscriptstyle j})$ was received in the time window: $(t_{\scriptscriptstyle j}, t_{\scriptscriptstyle j}+\Delta_{\scriptscriptstyle 1}+\Delta_{\scriptscriptstyle 2}]$ and whether $l_{\scriptscriptstyle j}$ was received in the time window: $(t'_{\scriptscriptstyle j}, t'_{\scriptscriptstyle j}+\Delta_{\scriptscriptstyle 2}]$

\item\label{check-hash}\textbf{\textit{\small {Verify Puzzle Solution}}}: runs $\mathtt{Verify}(.)$ in C-TLP to verify $\ddot{p}_{\scriptscriptstyle j}$  (i.e. checks the correctness of $l_{\scriptscriptstyle j}\in \ddot{p}_{\scriptscriptstyle j}$). If approved, then it regenerates the seed:  $u_{\scriptscriptstyle j}=\mathtt{H}( \mathcal {B}_{\scriptscriptstyle \gamma}||...||  \mathcal {B}_{\scriptscriptstyle \zeta})$, where $\gamma=w+(j-1)\cdot g$ and $\zeta=w+j\cdot g$


%$s_{\scriptscriptstyle j}=H( \mathcal {B}_{\resizeS {\textit  w}_{\resizeSS {\textit  j}}}||,...,||  \mathcal {B}_{\resizeS {\textit  w}_{\resizeSS {\textit  j}}+g})$, given parameters $g$ and $\resizeT{\textit{w}}_{\resizeS {\textit  j}}$. 


\item\label{verify-PoR}\textbf{\textit{\small {Verify PoR}}}: regenerates the pseudorandom values and verifies the PoR proof.  
\begin{equation}\label{POR-ver}\xi_{\scriptscriptstyle j}\stackrel{\scriptscriptstyle ?}=\mu_{\scriptscriptstyle j}  \cdot\mathtt{PRF}(l_{\scriptscriptstyle j},c+1)+\sum\limits^{\scriptscriptstyle c}_{\scriptscriptstyle b=1} ( \mathtt{PRF}(u_{\scriptscriptstyle j},b)\cdot \mathtt{PRF}(l_{\scriptscriptstyle j},b))\bmod p
\end{equation}
\item\textbf{\textit{\small {Pay}}}: if  Equation \ref{POR-ver} holds, pays and asks the server to delete all disposable tags for this verification, i.e. $\sigma_{\scriptscriptstyle j}$
\end{enumerate}
If either check fails, it aborts and notifies the client. 


\item \textit{\textbf{Client-server PoR}}: When the client is online, it can   interact  with the server  to check its data availability too. In particular, it sends $c$ random challenges and random indices to the server who computes POR using only: (a) the  messages sent by the client in this step, (b) the  file: ${\bm{F}}$, and (c) the tags:  $\sigma_{\scriptscriptstyle i}\in\sigma$, generated in step \ref{gen-client-server-tags}.  The proof generation and verification are similar to the MAC-based schemes, e.g.  \cite{DBLP:conf/asiacrypt/ShachamW08}. 
\end{enumerate}


\begin{theorem}\label{PoR-main-theorem} SO-PoR protocol is secure  if the MAC's are unforgeable, $\mathtt{PRF}(.)$ is a secure pseudorandom function, the blockchain and C-TLP protocol are secure, and $ \mathtt{H}( \mathcal {B}_{\scriptscriptstyle \gamma}||...||  \mathcal {B}_{\scriptscriptstyle \zeta})$ outputs an unpredictable random value (where $\zeta-\gamma$ is a security parameter).
\end{theorem}

\begin{remark}
We refer readers to Appendix \ref{SO-PoR-Security-Proof} for the above theorem's proof.
\end{remark}



\begin{remark}
In SO-PoR, for a security reason the server must record $j\text{\small{-th}}$ PoR proof in the contract before $l_{\scriptscriptstyle j}$ is recovered. Also,  the way disposable tags are generated in SO-PoR  differs from those computed  in previous PoR schemes, despite having similarities structure-wise. Moreover, with slight adjustments, we can reduce the contract-side storage cost to constant.  For more details, we refer readers to Appendix \ref{SO-PoR-discussion} which also explains why strawman approaches are not suitable substitutes for SO-PoR. 
\end{remark}
\vspace{-3mm}
%\subsection{Extension: Reducing Smart Contract Storage Cost to Constant}\label{storage-cost-reduction}
%
%
%With minor adjustments, we can reduce the smart contract storage cost from $O(z)$ to constant, $O(1)$ and offload the cost to the server. The idea is that the client after computing the commitmnet vector: $\vv{\bm{h}}=[h_{\scriptscriptstyle 1},...h_{\scriptscriptstyle z}]$,  in step \ref{Gen-Puzzles-}, it preserves the ordering of the elements (i.e. $h_{\scriptscriptstyle j}$ is associated with $j^{\scriptscriptstyle th} $ verification) and constructs a  Merkle tree  on top of them. It stores the tree and the vector on the server, and stores only the tree's  root: $R$, on the contract. In this case,  the server in step \ref{fully-recover-l} after recovering $\ddot{p}_{\scriptscriptstyle j}= (l_{\scriptscriptstyle j}, d_{\scriptscriptstyle j})$,  computes: $h_{\scriptscriptstyle j}=\mathtt{H}(l_{\scriptscriptstyle j}||d_{\scriptscriptstyle j})$, and sends a Merkle tree proof (that $h_{\scriptscriptstyle j}$ corresponds to  $R$) along with $\ddot{p}_{\scriptscriptstyle j}$ to the contract. In step \ref{check-hash}, the contract: (a) checks if $h_{\scriptscriptstyle j}=\mathtt{H}(l_{\scriptscriptstyle j}||d_{\scriptscriptstyle j})$, and  (b) verifies the Merkle tree proof.  The rest  remains unchanged.  As a result, the number of values stored in the contract is now $O(1)$. This adjustment comes with an added communication cost: $O(|h_{\scriptscriptstyle j}|\log z)$ for each verification. Nevertheless, the added cost is small and independent of the file size.   For instance, when  $z=10^{\scriptscriptstyle 6}$ and $|h_{\scriptscriptstyle j}|=256$, the  added communication cost is only about $5.1$ kilobit.

\vspace{-3mm}

\subsection{Evaluation}

In this section, we provide a summary of comparisons between   SO-PoR and outsourced PoRs  \cite{armknecht2014outsourced,xu2016lightweight,Storage-Time}. Among the two protocols proposed in \cite{Storage-Time} we only consider ``basic PoSt'' as it supports public verifiability. Briefly, in terms of property, only SO-PoR offers an explicit solution for real-time detection and fair payment. In terms of computation cost, the verification algorithm in SO-PoR is much faster than the other three protocols; Specifically, when $c=460$,  SO-PoR verification\footnote{As shown in \cite{DBLP:conf/ccs/AtenieseBCHKPS07}, to ensure $99\%$ of file blocks is retrievable, it  suffices to set $c=460$.} requires about $4.5$ times fewer computation than the verification required in the fastest outsourced PoR \cite{armknecht2014outsourced}.   Also, \cite{armknecht2014outsourced} has the worst store cost, which is  much higher than that of SO-PoR; e.g. for a $1$-GB file, SO-PoR requires over $46 \times 10^{\scriptscriptstyle 5}$ times fewer exponentiations than \cite{armknecht2014outsourced} needs in the same phase.  SO-PoR and \cite{Storage-Time} require a server to solve puzzles but the other two protocols do not need that. Also, the I/O cost and proof complexity of all protocols are $O(1)$ except \cite{Storage-Time} whose I/O cost and proof complexity are $O(\log n)$. The server-side bandwidth of SO-PoR is much lower than the rest;  for instance, for $1$-GB file and $z=100$ verifications, a server in SO-PoR requires $9\times 10^{\scriptscriptstyle4}$,  $7$ and $1729$ times fewer bits  than those required in \cite{armknecht2014outsourced}, \cite{xu2016lightweight} and \cite{Storage-Time} respectively.  A client in SO-PoR has a higher bandwidth than the rest of the protocols (but this cost is one-off). Thus, SO-PoR offers additional properties, it has lower verification cost and lower server-side bandwidth than the rest while its other costs remain reasonable. Table \ref{table::O-PoR-Cost} outlines the cost comparison results. For a  full analysis, we refer readers to Appendix \ref{Full-Evaluation} where  we also compare SO-PoR costs with the costs of the most efficient  traditional PoR  \cite{DBLP:conf/asiacrypt/ShachamW08}.


\vspace{-5mm}
% !TEX root =main.tex


 \begin{table*}[!htbp]

\caption{ \small Outsourced PoR's Cost  Comparison. In the table, $z$ is the total number of verifications, $c$ is  the number of challenges for each verification, $n$ is the total number of file blocks, $c'=(0.1)c$,  and $||\bm{F}||$ is a file bit size.} \label{table::O-PoR-Cost} 
\begin{footnotesize}
\begin{center}
\renewcommand{\arraystretch}{.80}
%\scalebox{1}{
%\begin{subtable}{.56\linewidth}%xxxx
\begin{minipage}{1\linewidth}
%\caption{\small Computation and I/O Costs}
\setlength{\tabcolsep}{1pt}%compress the table horizontally
\begin{tabular}{|c|c|c|c|c|c|c|c|c|c|c|c|c|c|c|c|} 

   \hline
\cellcolor[gray]{.9}&\cellcolor[gray]{.9}&
 \multicolumn{4}{c|}{\cellcolor[gray]{.9}\scriptsize \underline{ \  \  \  \  \   \  \    \  \  \  \ \  \  \  \  \  \  \ \  \  \      Computation Cost     \  \  \  \  \   \  \    \  \  \  \ \  \  \  \  \  \  \ \  \  \   }}& \multicolumn{4}{c|}{\cellcolor[gray]{.9}\scriptsize \underline{\  \  \  \  \  \  \ \ \  \  \  \  \  \   \  \  \  \  \  \  \  \  Communication Cost   \  \  \  \  \   \  \    \  \  \  \ \  \  \  \  \  \  \ \  \  \   }}\\
% \cline{3-6}
  \cellcolor[gray]{.9}\multirow{-2}{*}{\scriptsize Protocols} &\cellcolor[gray]{.9}\multirow{-2}{*} {\scriptsize Operation}&\cellcolor[gray]{.9}\scriptsize$\mathtt{Store}$&\cellcolor[gray]{.9}\scriptsize$\mathtt{SolvPuz}$&\cellcolor[gray]{.9}\scriptsize$\mathtt{Prove}$&\cellcolor[gray]{.9}\scriptsize$\mathtt{Verify}$&\cellcolor[gray]{.9} {\scriptsize Client}&\cellcolor[gray]{.9} {\scriptsize Server}&\cellcolor[gray]{.9} {\scriptsize Verifier}&\cellcolor[gray]{.9}{\scriptsize Proof Size}
  \\
\hline
    %SO-PoR 1st row
\cellcolor[gray]{.9}& \multirow{2}{*}{\rotatebox[origin=c]{0}{\cellcolor[gray]{.9}\scriptsize }} \scriptsize Exp.&\scriptsize$z+1$&\scriptsize$T z$&\scriptsize$-$ &\scriptsize$-$&\scriptsize $128(n+$&&&\\
     \cline{2-6}  
     %SO-PoR 2nd row
 \multirow{-2}{*}{\rotatebox[origin=c]{0}{\cellcolor[gray]{.9}\scriptsize  SO-PoR }}&\cellcolor[gray]{.9}\scriptsize Add. or Mul.&\scriptsize$2(n+cz)$ &\scriptsize$z$&\scriptsize$4 c z$&\scriptsize$2z(1+c)$&\scriptsize  $cz+19z)$&\multirow{-2}{*}{\scriptsize $884z$}&\multirow{-2}{*}{\scriptsize$-$}&\multirow{-2}{*}{\scriptsize $O(1)$}\\
     \cline{2-6}   
      
     \hline 
       %[3] 1st row 
       
          \hline 
          
 \cellcolor[gray]{.9}  &\multirow{2}{*}{\rotatebox[origin=c]{0}{\scriptsize }}\cellcolor[gray]{.9}\scriptsize Exp.&\scriptsize$9 n$&\scriptsize$ -$&\scriptsize$-$&$-$&&\scriptsize$256z+$&\scriptsize $4672n+$&\\
     \cline{2-6}
     %[3] 2nd row 
\cellcolor[gray]{.9}   \multirow{-2}{*}{\rotatebox[origin=c]{0}{\scriptsize   \cite{armknecht2014outsourced}}}  &\cellcolor[gray]{.9}\scriptsize Add. or Mul.&\scriptsize$10 n$&$-$&\scriptsize$4 z(c+c')$&\scriptsize$z(9c+3)$&\multirow{-2}{*} {\scriptsize $128n$}&\scriptsize $||{\bm{F}}||$&\scriptsize $256z$&\multirow{-2}{*} {\scriptsize $O(1)$}\\   
      
      \hline
      
       \hline
      %[53] 1st row
  \cellcolor[gray]{.9}    &\multirow{2}{*}{\rotatebox[origin=c]{0}{\  \scriptsize }}\cellcolor[gray]{.9}\scriptsize Exp.&\scriptsize$-$&$-$&\scriptsize$z (3+c)$&\scriptsize$6 z$&&&&\\
     \cline{2-6}
     %[53] 2nd row
 \cellcolor[gray]{.9}&\cellcolor[gray]{.9}\scriptsize Add. or Mul.&\scriptsize$4 n$&\scriptsize$-$&\scriptsize$2 z(3 c+4)$&\scriptsize$2 c z$&&&&\\ 
    \cline{2-6}
    %[53] 3rd row 
\cellcolor[gray]{.9}\multirow{-3}{*}{\rotatebox[origin=c]{0}{\scriptsize   \cite{xu2016lightweight}}}&\cellcolor[gray]{.9}\scriptsize Pairing&\scriptsize$-$&\scriptsize$-$&\scriptsize$7 z$&$-$&\multirow{-3}{*} {\scriptsize $2048n$}&\multirow{-3}{*} {\scriptsize $6144z$}&\multirow{-3}{*} {\scriptsize$-$}&\multirow{-3}{*} {\scriptsize $O(1)$}\\ 
 \hline
  
  %%%%%%%%%%%%%%%%%%%%%%%%%%%%%%
 
 \hline
\cellcolor[gray]{.9}& \multirow{2}{*}{\rotatebox[origin=c]{0}{\cellcolor[gray]{.9}\scriptsize }} \scriptsize Exp.&\scriptsize$-$&\scriptsize$3T z$&\scriptsize$-$ &\scriptsize$3z$&&\scriptsize$128cz \log n+$&&\\
     \cline{2-6}  
 \multirow{-2}{*}{\rotatebox[origin=c]{0}{\cellcolor[gray]{.9}\scriptsize  \cite{Storage-Time} }}&\cellcolor[gray]{.9}\scriptsize Add. or Mul.&\scriptsize$-$ &\scriptsize$Tz$&\scriptsize$-$&\scriptsize$-$&\multirow{-2}{*} {\scriptsize $128$}&\scriptsize $4096z$&\multirow{-2}{*} {\scriptsize$-$}&\multirow{-2}{*} {\scriptsize$O(\log n)$} \\
     \cline{2-6}    
     \hline 
 
 %%%%%%%%%%%%%%%%%%%%%%%%%%%%%%
 
 
 
\end{tabular}  %xxxxxx
\end{minipage}
%\end{center}
%\end{footnotesize}
%\end{table*}
%\end{subtable}


%\begin{subtable}{.52\linewidth}%xxxx
%\renewcommand{\arraystretch}{1.42}
%
%\begin{minipage}{.8\linewidth}
%\caption{\small Communication Cost (in bit)}
%\setlength{\tabcolsep}{.65pt}%compress the table horizontally
%
%\begin{tabular}{|c|c|c|c|c|c|} 
%   \hline
% \cellcolor[gray]{.9}\scriptsize Protocols&
%\cellcolor[gray]{.9}\scriptsize Client&\cellcolor[gray]{.9}\scriptsize Server&\cellcolor[gray]{.9}\scriptsize  Verifier&\cellcolor[gray]{.9}\scriptsize Proof Size\\
%   \hline
%   
%
%\cellcolor[gray]{.9}&\scriptsize$128 (n+$&\multirow{2}{*} {\scriptsize$884 z$} &\multirow{2}{*} {\scriptsize $-$}&\multirow{2}{*}{\scriptsize$O(1)$}\\  
%\cellcolor[gray]{.9}\multirow{-2}{*} {\scriptsize  SO-PoR}&\scriptsize$cz+19 z)$&&&\\
% \hline  
% 
%      \hline   
%\cellcolor[gray]{.9}& \multirow{2}{*} {\scriptsize$128 n$}&\multirow{2}{*} {\scriptsize$||\vv{\bm{F}}||+256 z$}&\scriptsize$4672 n+$&\multirow{2}{*}{\scriptsize$O(1)$}\\
%\cellcolor[gray]{.9}\multirow{-2}{*} {\scriptsize  \cite{armknecht2014outsourced}}&&&\scriptsize$256 z$&\\
% \hline
% 
%      \hline 
%\cellcolor[gray]{.9}\scriptsize\cite{xu2016lightweight}& \scriptsize$2048 n$&\scriptsize$6144z$&\scriptsize$-$&\scriptsize$O(1)$\\
% \hline
% 
%%%%%%%%%%%%%%%%%%%%
%
%
%      \hline 
%\cellcolor[gray]{.9}\scriptsize\cite{Storage-Time}& \scriptsize$128$&\scriptsize$(128cz\log n)+4096z$&\scriptsize$-$&\scriptsize$O(\log n)$\\
% \hline
%
%%%%%%%%%%%%%%%%%
% 
% 
%% \end{footnotesize}
%%\end{center} 
%\end{tabular}  %xxxxxx
%\end{minipage}
%\end{subtable}
%}
\end{center}
\end{footnotesize}
\end{table*}



%--------------------------------------------------------


% \begin{table*}[!htbp]
%\begin{footnotesize}
%\begin{center}
%\caption{ \small Outsourced PoR Cost  Comparison} \label{table::O-PoR-Cost} 
%\renewcommand{\arraystretch}{.80}
%%\scalebox{1}{
%%\begin{subtable}{.56\linewidth}%xxxx
%\begin{minipage}{1\linewidth}
%\caption{\small Computation and I/O Costs}
%\setlength{\tabcolsep}{.55pt}%compress the table horizontally
%\begin{tabular}{|c|c|c|c|c|c|c|c|c|c|c|c|c|c|c|c|} 
%
%   \hline
%\cellcolor[gray]{.9}&\cellcolor[gray]{.9}&
% \multicolumn{4}{c|}{\cellcolor[gray]{.9}\scriptsize \underline{\  \  \  \  \  \  \ \ \  \  \  \  \  \      Algorithms' Computation Cost     \  \  \  \ \  \  \  \  \  \  \ \  \  \   }}&\cellcolor[gray]{.9}&\cellcolor[gray]{.9}&\cellcolor[gray]{.9}&\cellcolor[gray]{.9}&\cellcolor[gray]{.9}\\
%% \cline{3-6}
%  \cellcolor[gray]{.9}\multirow{-2}{*}{\scriptsize Protocols} &\cellcolor[gray]{.9}\multirow{-2}{*} {\scriptsize Operation}&\cellcolor[gray]{.9}\scriptsize$\mathtt{Store}$&\cellcolor[gray]{.9}\scriptsize$\mathtt{SolvPuz}$&\cellcolor[gray]{.9}\scriptsize$\mathtt{Prove}$&\cellcolor[gray]{.9}\scriptsize$\mathtt{Verify}$&\cellcolor[gray]{.9}\multirow{-2}{*} {\scriptsize I/O Cost}&\cellcolor[gray]{.9}\multirow{-2}{*} {\scriptsize Client}&\cellcolor[gray]{.9}\multirow{-2}{*} {\scriptsize Server}&\cellcolor[gray]{.9}\multirow{-2}{*} {\scriptsize Verifier}&\cellcolor[gray]{.9}\multirow{-2}{*} {\scriptsize Proof Size}
%  \\
%\hline
%    %SO-PoR 1st row
%\cellcolor[gray]{.9}& \multirow{2}{*}{\rotatebox[origin=c]{0}{\cellcolor[gray]{.9}\scriptsize }} \scriptsize Exp.&\scriptsize$z+1$&\scriptsize$T z$&\scriptsize$-$ &$-$&&\scriptsize $128(n+$&&&\\
%     \cline{2-6}  
%     %SO-PoR 2nd row
% \multirow{-2}{*}{\rotatebox[origin=c]{0}{\cellcolor[gray]{.9}\scriptsize  SO-PoR }}&\cellcolor[gray]{.9}\scriptsize Add. or Mul.&\scriptsize$2(n+cz)$ &\scriptsize$z$&\scriptsize$4 c z$&\scriptsize$2z(1+c)$&\multirow{-2}{*} {\scriptsize $O(1)$}&\scriptsize  $cz+19z)$&\multirow{-2}{*}{\scriptsize $884z$}&\multirow{-2}{*}{\scriptsize -}&\multirow{-2}{*}{\scriptsize $O(1)$}\\
%     \cline{2-6}   
%      
%     \hline 
%       %[3] 1st row 
%       
%          \hline 
%          
% \cellcolor[gray]{.9}  &\multirow{2}{*}{\rotatebox[origin=c]{0}{\scriptsize }}\cellcolor[gray]{.9}\scriptsize Exp.&\scriptsize$9 n$&\scriptsize$ -$&\scriptsize$-$&$-$&&&\scriptsize$256z+$&\scriptsize $4672n+$&\\
%     \cline{2-6}
%     %[3] 2nd row 
%\cellcolor[gray]{.9}   \multirow{-2}{*}{\rotatebox[origin=c]{0}{\scriptsize   \cite{armknecht2014outsourced}}}  &\cellcolor[gray]{.9}\scriptsize Add. or Mul.&\scriptsize$10 n$&$-$&\scriptsize$4 z(c+c')$&\scriptsize$z(9c+3)$&\multirow{-2}{*} {\scriptsize $O(1)$}&\multirow{-2}{*} {\scriptsize $128n$}&\scriptsize $||\vv{\bm{F}}||$&\scriptsize $256z$&\multirow{-2}{*} {\scriptsize $O(1)$}\\   
%      
%      \hline
%      
%       \hline
%      %[53] 1st row
%  \cellcolor[gray]{.9}    &\multirow{2}{*}{\rotatebox[origin=c]{0}{\  \scriptsize }}\cellcolor[gray]{.9}\scriptsize Exp.&\scriptsize$-$&$-$&\scriptsize$z (3+c)$&\scriptsize$6 z$&&&&&\\
%     \cline{2-6}
%     %[53] 2nd row
% \cellcolor[gray]{.9}&\cellcolor[gray]{.9}\scriptsize Add. or Mul.&\scriptsize$4 n$&$-$&\scriptsize$2 z(3 c+4)$&\scriptsize$2 c z$&&&&&\\ 
%    \cline{2-6}
%    %[53] 3rd row 
%\cellcolor[gray]{.9}\multirow{-3}{*}{\rotatebox[origin=c]{0}{\scriptsize   \cite{xu2016lightweight}}}&\cellcolor[gray]{.9}\scriptsize Pairing&\scriptsize$-$&\scriptsize$-$&\scriptsize$7 z$&$-$&\multirow{-3}{*} {\scriptsize $O(1)$}&\multirow{-3}{*} {\scriptsize $2048n$}&\multirow{-3}{*} {\scriptsize $6144z$}&\multirow{-3}{*} {-}&\multirow{-3}{*} {\scriptsize $O(1)$}\\ 
% \hline
%  
%  %%%%%%%%%%%%%%%%%%%%%%%%%%%%%%
% 
% \hline
%\cellcolor[gray]{.9}& \multirow{2}{*}{\rotatebox[origin=c]{0}{\cellcolor[gray]{.9}\scriptsize }} \scriptsize Exp.&\scriptsize$-$&\scriptsize$3T z$&\scriptsize$-$ &\scriptsize$3z$&&&\scriptsize$128cz \log n+$&&\\
%     \cline{2-6}  
% \multirow{-2}{*}{\rotatebox[origin=c]{0}{\cellcolor[gray]{.9}\scriptsize  \cite{Storage-Time} }}&\cellcolor[gray]{.9}\scriptsize Add. or Mul.&\scriptsize$-$ &\scriptsize$Tz$&\scriptsize$-$&\scriptsize$-$&\multirow{-2}{*} {\scriptsize $O(\log n)$}&\multirow{-2}{*} {\scriptsize $128$}&\scriptsize $4096z$&\multirow{-2}{*} {-}&\multirow{-2}{*} {\scriptsize$O(\log n)$} \\
%     \cline{2-6}    
%     \hline 
% 
% %%%%%%%%%%%%%%%%%%%%%%%%%%%%%%
% 
% 
% 
%\end{tabular}  %xxxxxx
%\end{minipage}
%\end{center}
%\end{footnotesize}
%\end{table*}
%

%--------------------------------------------------------













%In general, the overall bandwidth of SO-PoR is much lower than \cite{armknecht2014outsourced}, and is about $9\times$ higher than the outsourced PoR that requires a \emph{trusted} verifier, i.e. \cite{xu2016lightweight}, due to  higher  client-side bandwidth.  Also, a client bandwidth   in SO-PoR requires $128(cz+19z)$  more bits than a client in the privately verifiable PoR \cite{DBLP:conf/asiacrypt/ShachamW08}, while the server's bandwidth in SO-PoR is $3.4$ times higher than that in \cite{DBLP:conf/asiacrypt/ShachamW08}.



